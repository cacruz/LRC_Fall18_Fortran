\documentclass[11pt]{beamer}
\setbeamertemplate{navigation symbols}{}
 \setbeamercovered{transparent}
\usepackage{listings}
%\usetheme{Copenhagen}
\usetheme{Singapore}
%\usetheme{Madrid}
%\usetheme{Hannover}
%\usetheme{boxes}
%\usetheme{Boadilla}
\usefonttheme[onlymath]{serif}
\usecolortheme{beaver}
\usepackage{textpos}
\usepackage{fancyvrb}
\usepackage{xcolor}
\usepackage{multicol}
\usepackage{lipsum}
\parskip 1ex

\newcommand\FontAcolumn{\fontsize{6}{7.2}\selectfont}
\newcommand\FontBcolumn{\fontsize{8}{7.2}\selectfont}
\newcommand\FontCcolumn{\fontsize{10}{7.2}\selectfont}
\newcommand\FontDcolumn{\fontsize{11}{7.2}\selectfont}

\definecolor{gray97}{gray}{.97}
\definecolor{gray75}{gray}{.75}
\definecolor{gray75}{gray}{.45}

\lstdefinestyle{Fortran}{language=[90]Fortran}

\newcommand\FortranStyle
{
\lstset{
frame=Ltb,
framerule=0pt,
columns=fullflexible,
aboveskip=0.5cm,
framextopmargin=3pt,
framexbottommargin=3pt,
framexleftmargin=0.4cm,
framesep=0pt,
rulesep=.4pt,
backgroundcolor=\color{gray97},
rulesepcolor=\color{black},
stringstyle=\ttfamily,
showstringspaces=false,
basicstyle=\ttfamily,
commentstyle=\color{green},
keywordstyle=\color{red},
numbers=left,
numbersep=15pt,
numberstyle=\tiny,
numberfirstline=false,
breaklines=true,
 tabsize=2,
 extendedchars=true,
keepspaces,
}
}

\newcommand\FortranStyleA
{
\lstset{
frame=Ltb,
framerule=0pt,
columns=fullflexible,
aboveskip=0.5cm,
framextopmargin=3pt,
framexbottommargin=3pt,
framexleftmargin=0.4cm,
framesep=0pt,
rulesep=.4pt,
backgroundcolor=\color{gray97},
rulesepcolor=\color{black},
stringstyle=\ttfamily,
showstringspaces=false,
basicstyle=\ttfamily,
commentstyle=\color{green},
keywordstyle=\color{red},
numbersep=15pt,
numberstyle=\tiny,
numberfirstline=false,
breaklines=true,
 tabsize=2,
 extendedchars=true,
keepspaces,
}
}

\newcommand\tab[1][1cm]{\hspace*{#1}}
\newcommand{\light}[1]{\textcolor{lightgray}{#1}}
    
\def\signed #1{{\leavevmode\unskip\nobreak\hfil\penalty50\hskip2em
  \hbox{}\nobreak\hfil(#1)%
  \parfillskip=0pt \finalhyphendemerits=0 \endgraf}}

\newsavebox\mybox
\newenvironment{aquote}[1]
  {\savebox\mybox{#1}\begin{quote}}
  {\signed{\usebox\mybox}\end{quote}}


% items enclosed in square brackets are optional; explanation below
\title{Miscellaneous Items}
\author{Carlos Cruz}
\institute{
  NASA GSFC Code 606 (ASTG)\\
  Greenbelt, Maryland 20771\\[1ex]
  \texttt{carlos.a.cruz@nasa.gov}
}
\date{October 25, 2018}

\begin{document}

% --- Title page ---
\begin{frame}[plain]
  \titlepage
\end{frame}

\logo{%
  \includegraphics[width=1cm,height=1cm,keepaspectratio]{../../shared/nasa-ball.png}%
  \hspace{\dimexpr\paperwidth-2cm-5pt}%
  \includegraphics[width=1cm,height=1cm,keepaspectratio]{../../shared/ssai-logo.png}%
}


% --- Slide

\begin{frame}{Agenda}

\textcolor{red}{Miscellaneous Items}
    \begin{itemize}
    \item Computing Environment
    \item New constructs
    \item Module Enhancements
    \begin{itemize}
    \item IMPORT
    \item New Attributes
    \item Renaming Operatos
    \end{itemize}
    \item Changes to Intrinsic Functions
    \item Complex Constants
    \end{itemize}
  

\end{frame}


% --- Slide ---

\begin{frame}{Accessing the Computing Environment}

\footnotesize{
\begin{itemize}
  \item For the following assume we have launched the executable with the command line: \\
  \quad \$ foo.x apple 5 z
  \item COMMAND\_ARGUMENT\_COUNT()
  \begin{itemize}
  \item Returns integer number of command arguments
  \item Example command returns 3
  \end{itemize}
 
 \item GET\_COMMAND([COMMAND,LENGTH,STATUS])
  \begin{itemize}
  \item All INTENT(OUT) and OPTIONAL
  \item LENGTH - integer \# of characters in command
  \item STATUS - integer (success/failure)
  \item Results for example command:
  \begin{itemize}
  \item COMMAND = "foo.x apple 5 z"
  \item LENGTH=15
  \end{itemize}
 \end{itemize}
\end{itemize}
}

\end{frame}


% --- Slide ---

\begin{frame}{Computing Environment}

\footnotesize{
\begin{itemize}
 \item GET\_COMMAND\_ARGUMENT(NUMBER[,VALUE,LENGTH,STATUS])
  \begin{itemize}
  \item NUMBER - selects argument
  \item VALUE - character, intent(out) value of argument
  \item LENGTH - number of characters in argument
  \item STATUS - integer (success/failure)
  \item Example command yields:
  \begin{itemize}
  \item GET\_COMMAND\_ARGUMENT(0,VALUE,LENGTH) yields \\
  \quad VALUE="foo.x", LENGTH=5
  \item GET\_COMMAND\_ARGUMENT(2,VALUE,LENGTH) yields \\
  \quad VALUE="5", LENGTH=1
  \end{itemize}
 \end{itemize}
\end{itemize}
}

\end{frame}

% --- Slide ---

\begin{frame}[fragile]
\frametitle{ISO\_FORTRAN\_ENV}

\footnotesize{
A new intrinsic module is ISO\_FORTRAN\_ENV. It contains the following constants
\begin{itemize}

 \item INPUT\_UNIT, OUTPUT\_UNIT, and ERROR\_UNIT 
  \begin{itemize}
  \item are default integer scalars holding the unit identified by an asterisk in a READ statement, an asterisk in a WRITE statement, and used for the purpose of error reporting, respectively.
  \end{itemize}
  
  \item IOSTAT\_END and IOSTAT\_EOR 
  \begin{itemize}
  \item are default integer scalars holding the values that are assigned to the IOSTAT= variable if an end-of-file or end-of-record condition occurs, respectively.
  \end{itemize}
  
  \item NUMERIC\_STORAGE\_SIZE, CHARACTER\_STORAGE\_SIZE, and FILE\_STORAGE\_SIZE 
  \begin{itemize}
   \item are default integer scalars holding the sizes in bits of a numeric, character, and file storage unit, respectively.
   \end{itemize}
   
\end{itemize}

}

\end{frame}


% --- Slide ---

\begin{frame}{Array Constructor}

\begin{itemize}
 \item Can now use "[" and "]" rather than "(/", "/)" to construct arrays:\\
 \quad x(1:5) = [0.,1.,2.,3.,4.]
  \item Can also specify type \textbf{inside} constructor
  \begin{itemize}
  \item VALUE - character, intent(out) value of argument
  \item LENGTH - number of characters in argument
  \item STATUS - integer (success/failure)
  \item Example command yields:
  \begin{itemize}
  \item GET\_COMMAND\_ARGUMENT(0,VALUE,LENGTH) yields \\
  \quad VALUE="foo.x", LENGTH=5
  \item GET\_COMMAND\_ARGUMENT(2,VALUE,LENGTH) yields \\
  \quad VALUE="5", LENGTH=1
  \end{itemize}
 \end{itemize}

\end{itemize}

\end{frame}


% --- Slide ---

\begin{frame}[fragile]
\frametitle{ASSOCIATE construct}
ASSOCIATE construct associates named entities with expressions or variables during the execution of its block.
\footnotesize{
\FortranStyle
\begin{lstlisting}[style=Fortran]
use constants, only: gas_constant
ASSOCIATE ( R=>gas_constant, T => temp, P=>press, V=>vol)
    P = n*R*T/V
END ASSOCIATE

ASSOCIATE ( Z => EXP(-(X**2+Y**2)) * COS(THETA) ) 
    Y = A*Z
END ASSOCIATE
\end{lstlisting}
}
\end{frame}


% --- Slide ---

\begin{frame}[fragile]
\frametitle{ALLOCATE statement}

\footnotesize{
The allocatable attribute is no longer restricted to arrays 
\FortranStyle
\begin{lstlisting}[style=Fortran]
type (matrix(m=10,n=20)) :: a
type (matrix(m=:,n=:)), allocatable :: b, c
ALLOCATE(b, source=a)
ALLOCATE(c, source=a)
\end{lstlisting}
allocates the scalar objects b and c to be 10 by 20 matrices with the value of a. 
}
\end{frame}

% --- Slide ---

\begin{frame}[fragile]
\frametitle{Transferring an allocation}


\footnotesize{
The intrinsic subroutine MOVE\_ALLOC(FROM,TO) has been introduced to move an allocation from one allocatable object to another. 
\FortranStyle
\begin{lstlisting}[style=Fortran]
REAL,ALLOCATABLE :: GRID(:),TEMPGRID(:)
...
ALLOCATE(GRID(-N:N) ! initial allocation of GRID 
...
ALLOCATE(TEMPGRID(-2*N:2*N)) ! allocate bigger grid 
TEMPGRID(::2) = GRID ! distribute values to new locations 
CALL MOVE_ALLOC(TO=GRID,FROM=TEMPGRID)
\end{lstlisting}
MOVE\_ALLOC provides a reallocation facility that avoids the problem that has beset all previous attempts: deciding how to spread the old data into the new object.
}
\end{frame}

% --- Slide ---

\begin{frame}[fragile]
\frametitle{SELECT TYPE construct}

\footnotesize{
The SELECT TYPE construct selects for execution at most one of its constituent blocks, depending on the dynamic type of a variable or an expression, known as the 'selector'. 
\FortranStyle
\begin{lstlisting}[style=Fortran]
CLASS matrix :: mat
...
SELECT TYPE (A => mat)
    TYPE IS (matrix)
       <code here>
    TYPE IS (sparse_matrix)
       <code here>
END SELECT
\end{lstlisting}
\begin{itemize}
\item The first block is executed if the dynamic type of \emph{mat} is \emph{matrix} and the second block is executed if it is \emph{sparse\_matrix}.  
\item The association of the selector \emph{mat} with its associate name \emph{A} is exactly as in an ASSOCIATE construct
\item In the second block, we may use \emph{A}  to access the extensions thus: \emph{A}\%sparse
\end{itemize}
}
\end{frame}


% --- Slide ---

\begin{frame}[fragile]
\frametitle{IMPORT statement}

\scriptsize{
A common pitfall when using F90/F95 is the declaration of an interface block that needs to "use" a derived type defined in the same module:
\FortranStyle
\begin{lstlisting}[style=Fortran]
module foo
   type bar
      integer :: I,J
   end type bar
   interface
      subroutine externFunc(B)
         use foo, only: bar ! Not allowed?
         type (bar) :: B
      end subroutine externFunc
  end interface
...
\end{lstlisting}
}

\end{frame}

% --- Slide ---

\begin{frame}[fragile]
\frametitle{IMPORT statement}

\scriptsize{
IMPORT is a new statement to address this issue.
  \begin{itemize}
  \item Very similar to USE statement.
  \item Specifies all entities in host scoping unit that are accessible
  \item \emph{Only} allowed in an interface body within a module
  \end{itemize}
 Example:
\FortranStyle
\begin{lstlisting}[style=Fortran]
...
interface
     subroutine externFunc(B)
        import foo, only: bar
        type (bar) :: B
     end subroutine externFunc
  end interface
\end{lstlisting}
}

\end{frame}

% --- Slide ---

\begin{frame}[fragile]
\frametitle{PROTECTED attribute}

\scriptsize{
F2003 introduces the new attribute PROTECTED which provides a safety mechanism analogous to INTENT(IN)
  \begin{itemize}
  \item Specifies that the variable (or pointer status) may be altered only within the host module.
  \item Property is recursive. I.e. if a variable of derived type is PROTECTED, all of its sub-objects also have the attribute
  \item For pointers, only the association status is protected. The target may be modified elsewhere.
  \end{itemize}
 Example:
\FortranStyle
\begin{lstlisting}[style=Fortran]
module foo
private ! Good default
real, public :: pi
protected :: pi ! Allow value to be read
...
\end{lstlisting}
}

\end{frame}

% --- Slide ---

\begin{frame}[fragile]
\frametitle{Renaming operators}

\begin{itemize}
 \item F2003 extends the rename capability on USE statements to include renaming operators that are not intrinsic operators:\\
 \FortranStyle
\begin{lstlisting}[style=Fortran]
USE a_mod, OPERATOR(.MyAdd.) => OPERATOR(.ADD.)
\end{lstlisting}
 \item This allows .MyAdd. to denote the operator .ADD. accessed from the module.
 \end{itemize}

\end{frame}


% --- Slide ---

\begin{frame}[fragile]
\frametitle{Changes to Intrinsic Functions}

\begin{itemize}
 \item Argument COUNT\_RATE for SYSTEM\_CLOCK() can now be of type real.
 \begin{itemize}
 \item Previously had to convert integer to compute reciprocal to determine elapsed time
 \end{itemize}
 \item  MAX, MAXLOC, MAXVAL, MIN, MINLOC, MINVAL have all been extend to apply to type CHARACTER
 \item ATAN2, LOG, and SQRT have minor changes to take into account positive/negative zero for vendors that support the distinction.
 \end{itemize}

\end{frame}

% --- Slide ---

\begin{frame}[fragile]
\frametitle{Lengths of Names/Constants}

\begin{itemize}
 \item Variables may be declared with names of up to 63 characters
 \item Statements of up to 256 lines are permitted.
 \item Primarily aimed at supporting automatic code generation
 \end{itemize}

\end{frame}

% --- Slide ---

\begin{frame}[fragile]
\frametitle{Complex Constants}

Named constants may be used to specify real or imaginary parts of a complex constant:
 \FortranStyle
\begin{lstlisting}[style=Fortran]
REAL, PARAMETER :: pi = 3.1415926535897932384 
COMPLEX :: C = (1.0, pi)
\end{lstlisting}

\end{frame}


\end{document}
