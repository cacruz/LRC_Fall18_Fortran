\documentclass[11pt]{beamer}
\setbeamertemplate{navigation symbols}{}
 \setbeamercovered{transparent}
\usepackage{listings}
%\usetheme{Copenhagen}
\usetheme{Singapore}
%\usetheme{Madrid}
%\usetheme{Hannover}
%\usetheme{boxes}
%\usetheme{Boadilla}
\usefonttheme[onlymath]{serif}
\usecolortheme{beaver}
\usepackage{textpos}
\usepackage{fancyvrb}
\usepackage{xcolor}
\usepackage{multicol}
\usepackage{lipsum}
\parskip 1ex

\newcommand\FontAcolumn{\fontsize{6}{7.2}\selectfont}
\newcommand\FontBcolumn{\fontsize{8}{7.2}\selectfont}
\newcommand\FontCcolumn{\fontsize{10}{7.2}\selectfont}
\newcommand\FontDcolumn{\fontsize{11}{7.2}\selectfont}

\definecolor{gray97}{gray}{.97}
\definecolor{gray75}{gray}{.75}
\definecolor{gray75}{gray}{.45}

\lstdefinestyle{Fortran}{language=[90]Fortran}

\newcommand\FortranStyle
{
\lstset{
frame=Ltb,
framerule=0pt,
columns=fullflexible,
aboveskip=0.5cm,
framextopmargin=3pt,
framexbottommargin=3pt,
framexleftmargin=0.4cm,
framesep=0pt,
rulesep=.4pt,
backgroundcolor=\color{gray97},
rulesepcolor=\color{black},
stringstyle=\ttfamily,
showstringspaces=false,
basicstyle=\ttfamily,
commentstyle=\color{green},
keywordstyle=\color{red},
numbers=left,
numbersep=15pt,
numberstyle=\tiny,
numberfirstline=false,
breaklines=true,
 tabsize=2,
 extendedchars=true,
keepspaces,
}
}

\newcommand\FortranStyleA
{
\lstset{
frame=Ltb,
framerule=0pt,
columns=fullflexible,
aboveskip=0.5cm,
framextopmargin=3pt,
framexbottommargin=3pt,
framexleftmargin=0.4cm,
framesep=0pt,
rulesep=.4pt,
backgroundcolor=\color{gray97},
rulesepcolor=\color{black},
stringstyle=\ttfamily,
showstringspaces=false,
basicstyle=\ttfamily,
commentstyle=\color{green},
keywordstyle=\color{red},
numbersep=15pt,
numberstyle=\tiny,
numberfirstline=false,
breaklines=true,
 tabsize=2,
 extendedchars=true,
keepspaces,
}
}

\newcommand\tab[1][1cm]{\hspace*{#1}}
\newcommand{\light}[1]{\textcolor{lightgray}{#1}}
    
\def\signed #1{{\leavevmode\unskip\nobreak\hfil\penalty50\hskip2em
  \hbox{}\nobreak\hfil(#1)%
  \parfillskip=0pt \finalhyphendemerits=0 \endgraf}}

\newsavebox\mybox
\newenvironment{aquote}[1]
  {\savebox\mybox{#1}\begin{quote}}
  {\signed{\usebox\mybox}\end{quote}}
  
% items enclosed in square brackets are optional; explanation below
\title{Introduction to Fortran}
%\subtitle{Introduction}
\author{Carlos Cruz}
\institute{
  NASA GSFC Code 606 (ASTG)\\
  Greenbelt, Maryland 20771\\[1ex]
  \texttt{carlos.a.cruz@nasa.gov}
}
\date{October 24, 2018}

\begin{document}

% --- Title page ---
\begin{frame}[plain]
  \titlepage
\end{frame}

\logo{%
  \includegraphics[width=1cm,height=1cm,keepaspectratio]{../../shared/nasa-ball.png}%
  \hspace{\dimexpr\paperwidth-2cm-5pt}%
  \includegraphics[width=1cm,height=1cm,keepaspectratio]{../../shared/ssai-logo.png}%
}

% --- Slide

\begin{frame}{Introduction to Fortran}

    \begin{itemize}
        \item History of Fortran
        \item Why Learn Fortran?
        \item First Look at Source Code
        \item Building Fortran Programs
    	\begin{itemize}
        		\item Compilers
        		\item Makefiles
   	 \end{itemize}
    \end{itemize}
 
\end{frame}

% --- Slide

\begin{frame}{A Brief History of Fortran}

\begin{itemize}
  \item Origins
  \begin{itemize}
  \scriptsize{
  \item Started ca 1954 by John Backus and his team at IBM
  \item Name comes from \textbf{FOR}mula \textbf{TRAN}slation
  \item First language standard in 1967 (Fortran 66)
  %Fortran 66 was the first ANSI standardized version of the language which made it portable. It introduced common data types, e.g. integer and double precision, block IF and DO statements
  }
  \end{itemize}
  
    \begin{figure}[t]
\centering
\includegraphics[scale=.3]{../../shared/fortran-card}
\end{figure}

  \item FORTRAN 77
  \begin{itemize}
  \scriptsize{
  \item New standard to overcome divergence in different implementations (1978)
  %Fortran 77 was also another major revision. It introduced file I/O and character data types
  }
  \end{itemize}
  
 
\end{itemize}

\end{frame}

% --- Slide

\begin{frame}{A Brief History of Fortran}

 
  
  \parindent0em


\begin{multicols}{2}[\columnsep2em] 


\begin{itemize}
  \item Fortran 90 (All caps were dropped)
  \begin{itemize}
  \scriptsize{
  \item Major revision. Added modules, derived data types, dynamic memory allocation
  \item Retained backward compatibility
  % Fortran 90 was a major step towards modernizing the language. It allowed free form code, array slicing, modules, interfaces and dynamic memory amongst other features 
  
  }
  \end{itemize}
  
  \item Fortran 95
  \begin{itemize}
  \scriptsize{
  \item Minor revision. Added several HPC related features; forall, where, pure,
  % Fortran 95 was a minor revision which includes pointers, pure and elemental features.
 elemental, pointers
  }
  \end{itemize}
  
\begin{figure}[t]
\centering
\includegraphics[scale=.3]{../../shared/Fortran-90-95}
\end{figure}
  
  
\end{itemize}


\columnbreak

\end{multicols}



 
\end{frame}

% --- Slide

\begin{frame}{A Brief History of Fortran}


\begin{columns}
  \begin{column}{0.48\textwidth}

\begin{itemize}
   
  \item Fortran 2003
  \begin{itemize}
  \scriptsize{
  \item Major revision with many new features including; OO capabilities, procedure pointers, IEEE arithmetic, C interoperability
  }
  \end{itemize}
  \item Fortran 2008 (latest \emph{stable} release)
  \begin{itemize}
  \scriptsize{
  \item Minor revision. Added co-arrays and submodules  
  }
  \end{itemize}
\end{itemize}

  \end{column}
  \begin{column}{0.48\textwidth}

\begin{figure}[t]
\centering
\includegraphics[scale=.3]{../../shared/oop.png}
\end{figure}
  

  \end{column}
\end{columns}

\end{frame}

% --- Slide ---
\begin{frame}[fragile]
\frametitle{Why Fortran?}

\begin{itemize}
\item Fortran is the dominant language in HPC applications
 \begin{itemize}
 \item climate/weather models
 \item large scale molecular dynamics
 \item electronic structure calculation codes
 \item modeling of stars and galaxies
 \end{itemize}
 \item Many dense linear algebra libraries developed in Fortran, e.g. BLAS, LAPACK, Scalapack.
\end{itemize}

Fortran continues and will continue to be a dominant language for large scale simulation of 
physical systems\footnote{Survey of Fortran users at the 2014 Supercomputing Convention, 100\% of respondents said they thought they would still be using Fortran in five years}.\\

%\scriptsize{
%Reference:\\
%Fortran: The Ideal HPC Programming Language: \url{https://queue.acm.org/detail.cfm?id=1820518}
%}
\end{frame}




% --- Slide

\begin{frame}{Why Fortran?}
% https://arstechnica.com/science/2014/05/scientific-computings-future-can-any-coding-language-top-a-1950s-behemoth/

Unique benefits:
\begin{itemize}

\item Expressiveness / ease
 \begin{itemize}
 \item Arrays lie at the heart of all physics/engineering calculations
 % The array is the most common data structure in computational science.
 \item Little to worry about pointers and memory allocation
 \begin{itemize}
 \item Dynamic arrays in Fortran are not pointers, where in C/C++ they are pointers making them more difficult to deal with.
 \end{itemize}
 \end{itemize}
\item Performance\footnote{Originally proposed as a practical alternative to \textbf{Assembly} language, i.e. it was designed to compete on performance with hand-coded assembler!} 
\item Has added many modern features of programming into newer standards.
\item Legacy code
\item It is \textbf{not} a general purpose programming language like C, C++ and Python.
 \begin{itemize}
 \item It is a language that has been designed exclusively for numerical computation and has applications only in computational science and engineering.
 \end{itemize}
 \end{itemize}
% Other reasons:
% generic names for procedures, op args
% recursive proc, op overloading
% struct data or deriv types
% oop

\end{frame}

% --- Slide ---
\begin{frame}[fragile]
\frametitle{}

\begin{center}
\huge{
General Structure of Fortran Programs
}
\end{center}

\end{frame}

% --- Slide ---

\begin{frame}{Source Code}

\begin{itemize}
\item Fortran source code is in ASCII text that can be written in any plain-text editor
\begin{itemize}
	\item Use a good editor, e.g. vim or emacs - both are capable of syntax highlighting.
\end{itemize}
\item Fortran source code is \textbf{case insensitive}, i.e. \emph{X} is the same \emph{x}
\end{itemize}

\end{frame}

% --- Slide

\begin{frame}[fragile]
\frametitle{Basic Program Structure}

 \footnotesize{The general structure of a Fortran \textbf{program} is as follows:}
\begin{columns}
 \begin{column}{0.48\textwidth}
 \footnotesize{
\FortranStyle
 \begin{lstlisting}[style=Fortran]
! keyword PROGRAM
PROGRAM program_name 
!  All variable MUST be declared
IMPLICIT NONE
! Variable declarations
 ...   type declarations 
! Main program
 ...    Fortran statements here
! subprograms
 ...    optional subprogram units
END PROGRAM
\end{lstlisting}
}
  \end{column}
  \begin{column}{0.48\textwidth}
\begin{figure}[t]
\centering
\includegraphics[scale=.3]{../../shared/main}
\end{figure}

  \end{column}
\end{columns}

 \footnotesize{The unit containing the PROGRAM attribute is often called the \textbf{main program} or \textbf{main}.}
\end{frame}

% --- Slide

\begin{frame}[fragile]
\frametitle{Hello World!}

\FortranStyle

hello.F90:
 \begin{lstlisting}[style=Fortran]
program hello
! A simple hello world program
   print *,"Hello World!"
end program
\end{lstlisting}

The standard extension for Fortran source files is .f90, i.e., the
source files are named $<$name$>$.f90. But:
\scriptsize{
\begin{itemize}
\item FORTRAN77:
\begin{itemize}
	\item .f   $\leftarrow$ \emph{fixed-form} source code
\end{itemize}
\item Fortran90 and later:
\begin{itemize}
	\item * .F $\leftarrow$ \emph{fixed or free form} source code that must be preprocessed
	\item * .F90 $\leftarrow$ \emph{free form} source code that must be preprocessed
\end{itemize}
\end{itemize}
}

\end{frame}


% --- Slide ---
\begin{frame}[fragile]
\frametitle{}


\begin{center}
\huge{
Compiling, Linking and Running Fortran Programs
}
\end{center}


\end{frame}



% --- Slide ---

\begin{frame}[fragile]
\frametitle{Compilers}

To execute a Fortran program, you need to \textbf{compile} it to obtain an executable.

\vfill
A \textbf{compiler} is a program that translates source code written in some high-level language - like Fortran - into low-level machine code. The machine code is generally specific to the computer architecture where it was compiled.

% refers to the processing of source code files
%and the creation of an 'object' file
%produces the machine language instructions that correspond to the source code file

\end{frame}



% --- Slide ---

\begin{frame}{Compilation at a glance}

How does a compiler translate source code into executable code?

\begin{figure}[t]
\centering
\includegraphics[scale=.3]{../../shared/compiler}
\end{figure}

\end{frame}


\begin{frame}{Some Fortran compiler vendors}

\begin{itemize}
\item \textcolor{red}{GNU}
\item IBM
\item Intel
\item NAG
\item PGI
\end{itemize}

\scriptsize{
\url{https://en.wikipedia.org/wiki/List_of_compilers\#Fortran_compilers}
}

\end{frame}


% --- Slide

\begin{frame}{The GNU Fortran compiler}

The GNU Fortran compiler supports the Fortran 77, 90 and 95 standards completely, parts
of the Fortran 2003 and Fortran 2008 standards, and several vendor extensions. \\
The GNU Fortran compiler has several components:
\begin{itemize}
\item A version of the \emph{gcc} command that also understands and accepts Fortran source code.
\item The \emph{gfortran} command. The difference with \emph{gcc} is that \emph{gfortran} will automatically link the correct libraries to your program.
\item A collection of run-time libraries. 
\item The Fortran compiler itself, \emph{f951}. \emph{f951} \textbf{translates} the source code to assembler code.
 \end{itemize}
GNU Fortran is a part of GCC, the \emph{GNU Compiler Collection}.

\end{frame}


% --- Slide ---

\begin{frame}{Using the GNU compiler}

Use \emph{gfortran} command:

\begin{itemize}
\item gfortran hello.F90  $\leftarrow$ no \emph{output} name for the executable specified
\begin{itemize}
\item ./a.out $\leftarrow$ default output file name
\end{itemize}

\item \light{gfortran -c hello.F90 $\leftarrow$ compile only - generates \emph{object} file}
\begin{itemize}
\item \light{hello.o}
\end{itemize}

\item \light{gfortran -c hello.F90 -o hello.o $\leftarrow$  same as previous case}
\begin{itemize}
\item \light{hello.o}
\end{itemize}

\item \light{gfortran -o hello.exe hello.F90 $\leftarrow$  compile and link}
\begin{itemize}
\item \light{./hello.exe}
\end{itemize}

\end{itemize}

\end{frame}

% --- Slide ---

\begin{frame}{Using the GNU compiler}

Use \emph{gfortran} command:

\begin{itemize}
\item \light{gfortran hello.F90  $\leftarrow$ no \emph{output} name for the executable specified}
\begin{itemize}
\item \light{./a.out $\leftarrow$ default output file name}
\end{itemize}

\item gfortran -c hello.F90 $\leftarrow$ compile only - generates \emph{object} file
\begin{itemize}
\item hello.o
\end{itemize}

\item gfortran -c hello.F90 -o hello.o $\leftarrow$  same as previous case
\begin{itemize}
\item hello.o
\end{itemize}

\item \light{gfortran -o hello.exe hello.F90 $\leftarrow$  compile and link}
\begin{itemize}
\item \light{./hello.exe}
\end{itemize}

\end{itemize}

\end{frame}


% --- Slide ---

\begin{frame}{Using the GNU compiler}

Use \emph{gfortran} command:

\begin{itemize}
\item \light{gfortran hello.F90  $\leftarrow$ no \emph{output} name for the executable specified}
\begin{itemize}
\item \light{./a.out $\leftarrow$ default output file name}
\end{itemize}

\item \light{gfortran -c hello.F90 $\leftarrow$ compile only - generates \emph{object} file}
\begin{itemize}
\item \light{hello.o}
\end{itemize}

\item \light{gfortran -c hello.F90 -o hello.o $\leftarrow$  same as previous case}
\begin{itemize}
\item \light{hello.o}
\end{itemize}

\item gfortran -o hello.exe hello.F90 $\leftarrow$  compile and link
\begin{itemize}
\item ./hello.exe
\end{itemize}

\end{itemize}

\end{frame}

% --- Slide ---

\begin{frame}[fragile]
\frametitle{Libraries}

Generally, we will have many fortran files that, upon compilation, will produce many object files. It is often useful to combine the object files into a \emph{library}. To create a \emph{static} library we use the program�"ar"�(stands for archiver) as follows:
\begin{Verbatim} 
$ gfortran -c *.F90
$ ar -r libmystatlib.a *.o
\end{Verbatim}
one can also create a \emph{dynamic} library...
\begin{Verbatim}
$ gfortran -c -fPIC *.F90  
$ gfortran -shared -o libmydynlib.so *.o
\end{Verbatim}

\end{frame}

% --- Slide ---

\begin{frame}[fragile]
\frametitle{make}
What is \textbf{make}?
\begin{itemize}
\item The \textbf{make} utility is a tool for managing and maintaining computer programs. It is generally used to build executable programs, libraries, and other files.
\item Created at Bell Labs ca. 1977 
\item The input to make is a file called the \textbf{makefile}.
\item The \textbf{makefile}  describes a set of targets, which are the objects that can be \textbf{made}.
\item \textbf{make} is programming language agnostic.
\item \textbf{make} is one of the most essential tools for programmers.
\end{itemize}
Reference: \url{http://www.gnu.org/software/make/}

\end{frame}

% Make is the de-facto tool for building executable programs from source code in the world of open source.
% Make enables end users to build executable programs without knowing technical details of how to build them

% --- Slide ---

\begin{frame}[fragile]
\frametitle{The makefile}

\scriptsize{
The \textbf{makefile} is a collection of \emph{rules}. 
\begin{itemize}
\item The rules instruct \textbf{make} how to build a target 
\item A rule also specifies dependencies of the target. 
\end{itemize}
The dependency rules must be executed first depending on whether that is already 
processed by looking at the time stamps.\\
\vspace{5mm}
Rule syntax:\\
\vspace{5mm}
\emph{target}: \emph{dependencies}\quad \quad $\leftarrow$ AKA a \emph{prerequisite}\\
\tab \emph{commands}\\
\quad $\uparrow$\\
Tab
}
\end{frame}

%The beauty of Make is that it?s simply a rigorous way of recording what you?re already doing. It doesn?t fundamentally change how you do something, but it encourages to you record each step in the process, enabling you (and your coworkers) to reproduce the entire process later.

%The core concept is that generated files depend on other files. When generated files are missing, or when files they depend on have changed, needed files are re-made using a sequence of commands you specify. 


% --- Slide ---

\begin{frame}[fragile]
\frametitle{The makefile}
\begin{itemize}
\item Standard names: makefile, Makefile, GNUmakefile.
\item To run:
\begin{Verbatim}
$ make  
$ make some_target
\end{Verbatim}
\item If using a non-standard name, say my\_app.make
\begin{Verbatim}
$ make -f my_app.make
\end{Verbatim}
\end{itemize}

\end{frame}


% --- Slide ---

\begin{frame}[fragile]
\frametitle{The makefile}

\footnotesize{
\begin{Verbatim}[frame=single]
hello.exe: hello.o
        gfortran hello.o -o hello.exe 
        
hello.o: hello.F90
        gfortran -c hello.F90
clean:
	rm *.o *.exe
\end{Verbatim}

\begin{verbatim}
$ make  (or make hello.exe)
gfortran -o hello.exe hello.F90 -I.
$ ls
GNUmakefile  hello.exe  hello.F90
$ ./hello.exe
 hello world!
\end{verbatim}
}

\end{frame}


% --- Slide

\begin{frame}[fragile]
\frametitle{Conclusion}

References:
\scriptsize{
\begin{itemize}
\item \url{https://gcc.gnu.org/fortran/}
\item \url{http://fortranwiki.org/fortran/show/HomePage}
\item \url{https://en.wikipedia.org/wiki/Fortran}
\item \url{https://www.gnu.org/software/make/manual/make.html}
\end{itemize}
}

\end{frame}

\end{document}
