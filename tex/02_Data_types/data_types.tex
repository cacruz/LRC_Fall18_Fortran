\documentclass[11pt]{beamer}
\setbeamertemplate{navigation symbols}{}
 \setbeamercovered{transparent}
\usepackage{listings}
%\usetheme{Copenhagen}
\usetheme{Singapore}
%\usetheme{Madrid}
%\usetheme{Hannover}
%\usetheme{boxes}
%\usetheme{Boadilla}
\usefonttheme[onlymath]{serif}
\usecolortheme{beaver}
\usepackage{textpos}
\usepackage{fancyvrb}
\usepackage{xcolor}
\usepackage{multicol}
\usepackage{lipsum}
\parskip 1ex

\newcommand\FontAcolumn{\fontsize{6}{7.2}\selectfont}
\newcommand\FontBcolumn{\fontsize{8}{7.2}\selectfont}
\newcommand\FontCcolumn{\fontsize{10}{7.2}\selectfont}
\newcommand\FontDcolumn{\fontsize{11}{7.2}\selectfont}

\definecolor{gray97}{gray}{.97}
\definecolor{gray75}{gray}{.75}
\definecolor{gray75}{gray}{.45}

%%%%%%%%%%%%%%%%%%%%%%%%%%%%%%%%%%%%%%%%%%%%%%%%%%%%%%%%
%%%% This is used to add line number to source code %%%%
%%%%%%%%%%%%%%%%%%%%%%%%%%%%%%%%%%%%%%%%%%%%%%%%%%%%%%%%
\lstset{
        language=fortran,        % language of the code
        basicstyle=\small\ttfamily, % size of the fonts used for the code
        %basicstyle=\normalsize, % size of the fonts used for the code
        numbers=left,           % where to put line numbers
        numberstyle=\tiny,     % size of the fonts used for line numbers
        stepnumber=1, 
        numbersep=8pt,
        numberfirstline=false,
        showstringspaces=false, % underline spaces within strings
        aboveskip=-20pt,
        numbersep=15pt,          % how far the line-numbers are from the code
        frame=Ltb,         % addition of a left frame line on source
        % code
        framerule=0pt,
        columns=fullflexible,
        framextopmargin=3pt,
        framexbottommargin=3pt,
        framexleftmargin=0.4cm,
        framesep=0pt,
        rulesep=.4pt,
        backgroundcolor=\color{gray97},
        rulesepcolor=\color{black},
        stringstyle=\ttfamily,
        showstringspaces=false,
        basicstyle=\ttfamily,
        keywordstyle=\color{red}, % color keywords
        commentstyle=\color{green}    % color comments
        }
\lstdefinestyle{Fortran}{language=[90]Fortran}

\newcommand\FortranStyle
{
\lstset{
frame=Ltb,
framerule=0pt,
columns=fullflexible,
aboveskip=0.5cm,
framextopmargin=3pt,
framexbottommargin=3pt,
framexleftmargin=0.4cm,
framesep=0pt,
rulesep=.4pt,
backgroundcolor=\color{gray97},
rulesepcolor=\color{black},
stringstyle=\ttfamily,
showstringspaces=false,
basicstyle=\ttfamily,
commentstyle=\color{green},
keywordstyle=\color{red},
numbers=left,
numbersep=15pt,
numberstyle=\tiny,
numberfirstline=false,
breaklines=true,
 tabsize=2,
 extendedchars=true,
keepspaces,
}
}

\newcommand\FortranStyleA
{
\lstset{
frame=Ltb,
framerule=0pt,
columns=fullflexible,
aboveskip=0.5cm,
framextopmargin=3pt,
framexbottommargin=3pt,
framexleftmargin=0.4cm,
framesep=0pt,
rulesep=.4pt,
backgroundcolor=\color{gray97},
rulesepcolor=\color{black},
stringstyle=\ttfamily,
showstringspaces=false,
basicstyle=\ttfamily,
commentstyle=\color{green},
keywordstyle=\color{red},
numbersep=15pt,
numberstyle=\tiny,
numberfirstline=false,
breaklines=true,
 tabsize=2,
 extendedchars=true,
keepspaces,
}
}

\newcommand\tab[1][1cm]{\hspace*{#1}}
\newcommand{\light}[1]{\textcolor{lightgray}{#1}}
    
\def\signed #1{{\leavevmode\unskip\nobreak\hfil\penalty50\hskip2em
  \hbox{}\nobreak\hfil(#1)%
  \parfillskip=0pt \finalhyphendemerits=0 \endgraf}}

\newsavebox\mybox
\newenvironment{aquote}[1]
  {\savebox\mybox{#1}\begin{quote}}
  {\signed{\usebox\mybox}\end{quote}}

  
% items enclosed in square brackets are optional; explanation below
\title{Variables and Data Types}
%\subtitle{Introduction}
\author{Carlos Cruz\\
Jules Kouatchou\\
Bruce Van Aartsen}
\institute{
  NASA GSFC Code 606 (ASTG)\\
  Greenbelt, Maryland 20771\\[1ex]
%  \texttt{carlos.a.cruz@nasa.gov}
}
\date{October 24, 2018}

\begin{document}

% --- Title page ---
\begin{frame}[plain]
  \titlepage
\end{frame}

\logo{%
  \includegraphics[width=1cm,height=1cm,keepaspectratio]{../../shared/nasa-ball.png}%
  \hspace{\dimexpr\paperwidth-2cm-5pt}%
  \includegraphics[width=1cm,height=1cm,keepaspectratio]{../../shared/ssai-logo.png}%
}

% --- Slide

\frame[containsverbatim]{
  \frametitle{Variables}
\begin{itemize}
\item Variables are used to hold values and then write mathematical expressions using them
\item A variable can hold one value at a time.
\item Variables must be declared at the start of the program.
\item Each variable must be named. The variable name is how variables are referred to by the program.
 \item A variable name must start with a letter, followed by letters,
   numbers, or an underscore.
 \item A variable name may not be longer than 32 characters.
  \item Capital letters are treated the same way as lower-case letters
\end{itemize}

}

\frame[containsverbatim]{
  \frametitle{Examples of Variable Names}
 Some valid variable names:
\begin{verbatim}
 x
 today
 next_month
 summation10
\end{verbatim}
 Some invalid variable names:
\begin{verbatim}
 1today
 this_is_a_variable_name_with_way_way_too_many_characters_in_it
 next@month
 next month
 today!
\end{verbatim}
}

 \frame[containsverbatim]{
  \frametitle{Variable Declaration }
  \begin{semiverbatim}
  \begin{lstlisting}
   <type> :: <list of variable names>
  \end{lstlisting}
\end{semiverbatim}
The type must be one of the predefined data types.
}

\frame[containsverbatim]{
  \frametitle{Data Types}
Fortran has five intrinsic data types:
\begin{itemize}
\item Integer
\item Real
\item Complex
\item Logical
 \item Character
\end{itemize}
You can derive your own data types as well.
}

\frame[containsverbatim]{
  \frametitle{Integer Data Type}
\begin{itemize}
\item Can hold only integer values (whole numbers)
\item You can specify the number of bytes using the \textit{kind}
  specifier
\item For integer division, no rounding will occur as the fractional part is truncated
\end{itemize}

}



\frame[containsverbatim]{
  \frametitle{Examples of Integer Data Type}
  \begin{semiverbatim}
  \begin{lstlisting}
program testingInteger
implicit none

   integer(kind =  2) :: shortval          !two byte integer
   integer(kind =  4) :: longval           !four byte integer
   integer(kind =  8) :: verylongval       ! eight byte integer
   integer(kind = 16) :: veryverylongval   ! sixteen byte integer
   integer            :: defval            ! default integer 
        
   print *, huge(shortval)
   print *, huge(longval)
   print *, huge(verylongval)
   print *, huge(veryverylongval)
   print *, huge(defval)
end program testingInteger
  \end{lstlisting}
  \end{semiverbatim}
}

\frame[containsverbatim]{
  \frametitle{Real Data Type}
\begin{itemize}
\item Also referred to as floating-point numbers, include both rational
numbers and irrational numbers
\item There are two different real types, the default \textit{real} type and \textit{double precision} type
\item You can specify the precision of real using the \textit{kind} specifier
\end{itemize}
}

\frame[containsverbatim]{
  \frametitle{Examples of Real Data Type}
  \begin{semiverbatim}
  \begin{lstlisting}
program testingReal   
implicit none  
   real(kind=4):: real4Res  ! Define single precision real variable
   real(kind=8):: real8Res ! Define double precision real variable
   integer ::  intRes    ! Define integer variables
   
   ! floating point division
   real4Res = 2.0/3.0
   real8Res = 2.0/3.0 
   intRes = 2/3
   
   print *, 'Single precision division:', real4Res
   print *, 'Double precision division:', real8Res
   print *, 'Integer division:', intRes
end program testingReal 
  \end{lstlisting}
  \end{semiverbatim}
}


\frame[containsverbatim]{
  \frametitle{Complex Type}
\begin{itemize}
\item Used to store complex number (number comprising a real number
  and an imaginary number).
 \item Not used extensively, but can be useful when needed.
\end{itemize}
}

\frame[containsverbatim]{
  \frametitle{Example of Complex Type}
  \begin{semiverbatim}
  \begin{lstlisting}
program testingComplex   
implicit none  
   real :: x, y
   complex :: cx1, cx2, cx3

   x   = 2.67
   y   = -0.349
   cx1 = (3.0, 5.0)            ! cx1 = 3.0 + 5.0i
   cx2 = cmplx (1.0/2.0, -7.0) ! cx2 = 0.5 – 7.0i 
   cx3 = cmplx (x, y)          ! cx3 = x + yi
   print *, 'cx1:     ', cx1
   print *, 'cx2:     ', cx2
   print *, 'cx3:     ', cx3
   print *, 'cx1+cx3: ', cx1+cx3
   print *, 'cx1-cx2: ', cx1-cx2
   print *, 'cx1*cx2: ', cx1*cx2
   print *, 'cx1/cx2: ', cx1/cx2
end program testingComplex 
  \end{lstlisting}
  \end{semiverbatim}
}


\frame[containsverbatim]{
  \frametitle{Logical Type}
  \begin{itemize}
   \item Stores logical variable
\item Takes two possible values: \textit{.true.} or \textit{.false.}.
\end{itemize}
}

\frame[containsverbatim]{
  \frametitle{Example of Logical Type}
\begin{semiverbatim}
  \begin{lstlisting}
program testingLogical   
implicit none  
INTEGER :: YEAR
LOGICAL :: LEAP_FLAG

YEAR = 2004
LEAP_FLAG = .FALSE.
IF (MOD(YEAR,4) .EQ. 0)   LEAP_FLAG = .TRUE.
IF (MOD(YEAR,100) .EQ. 0) LEAP_FLAG = .FALSE.
IF (MOD(YEAR,400) .EQ. 0) LEAP_FLAG = .TRUE.

PRINT*, 'Is ', YEAR, ' a leap year? ', LEAP_FLAG
end program testingLogical 
  \end{lstlisting}
  \end{semiverbatim}
}

\frame[containsverbatim]{
  \frametitle{Character Type}
\begin{itemize}
\item Stores a character (symbol like a letter, numerical digit, or
  punctuation) and a string (sequence or set of characters).
  \item Characters and strings are typically enclosed in quotes.
  \item The length of the string can be specified by \textit{len}
    specifier
\item If no length is specified, then \textit{len=1}.
\end{itemize}
}

\frame[containsverbatim]{
  \frametitle{Examples of Character Data Type}
  \begin{semiverbatim}
  \begin{lstlisting}
program testingCharacter
implicit none
integer, parameter :: maxLengthChar = 50
character (len = maxLengthChar) :: stationName
character (len = maxLengthChar) :: welcomeMsg
character (len = maxLengthChar) :: location
character                       :: oneChar

welcomeMsg = '       Welcome to the Fortran Tutorial    '
location = 'Hampton, Virginia''
stationName = 'Blodgett'
oneChar     = 'T'
print *, 'Message - 1:     ', welcomeMsg
print *, 'Message - 2:     ', TRIM(welcomeMsg)
print *, 'Message - 3:     ', TRIM(welcomeMsg)//' in '//TRIM(location)
print *, 'Name of station: ', stationName
print *, oneChar
end program testingCharacter
  \end{lstlisting}
  \end{semiverbatim}
}



\end{document}

