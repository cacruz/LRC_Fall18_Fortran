\documentclass[11pt]{beamer}
\setbeamertemplate{navigation symbols}{}
 \setbeamercovered{transparent}
\usepackage{listings}
%\usetheme{Copenhagen}
\usetheme{Singapore}
%\usetheme{Madrid}
%\usetheme{Hannover}
%\usetheme{boxes}
%\usetheme{Boadilla}
\usefonttheme[onlymath]{serif}
\usecolortheme{beaver}
\usepackage{textpos}
\usepackage{fancyvrb}
\usepackage{xcolor}
\usepackage{multicol}
\usepackage{lipsum}
\parskip 1ex

\newcommand\FontAcolumn{\fontsize{6}{7.2}\selectfont}
\newcommand\FontBcolumn{\fontsize{8}{7.2}\selectfont}
\newcommand\FontCcolumn{\fontsize{10}{7.2}\selectfont}
\newcommand\FontDcolumn{\fontsize{11}{7.2}\selectfont}

\definecolor{gray97}{gray}{.97}
\definecolor{gray75}{gray}{.75}
\definecolor{gray75}{gray}{.45}

%%%%%%%%%%%%%%%%%%%%%%%%%%%%%%%%%%%%%%%%%%%%%%%%%%%%%%%%
%%%% This is used to add line number to source code %%%%
%%%%%%%%%%%%%%%%%%%%%%%%%%%%%%%%%%%%%%%%%%%%%%%%%%%%%%%%
\lstset{
        language=fortran,        % language of the code
        basicstyle=\small\ttfamily, % size of the fonts used for the code
        %basicstyle=\normalsize, % size of the fonts used for the code
        numbers=left,           % where to put line numbers
        numberstyle=\tiny,     % size of the fonts used for line numbers
        stepnumber=1, 
        numbersep=8pt,
        numberfirstline=false,
        showstringspaces=false, % underline spaces within strings
        aboveskip=-20pt,
        numbersep=15pt,          % how far the line-numbers are from the code
        frame=Ltb,         % addition of a left frame line on source
        % code
        framerule=0pt,
        columns=fullflexible,
        framextopmargin=3pt,
        framexbottommargin=3pt,
        framexleftmargin=0.4cm,
        framesep=0pt,
        rulesep=.4pt,
        backgroundcolor=\color{gray97},
        rulesepcolor=\color{black},
        stringstyle=\ttfamily,
        showstringspaces=false,
        basicstyle=\ttfamily,
        keywordstyle=\color{red}, % color keywords
        commentstyle=\color{green}    % color comments
        }


\lstdefinestyle{Fortran}{language=[90]Fortran}

\newcommand\FortranStyle
{
\lstset{
frame=Ltb,
framerule=0pt,
columns=fullflexible,
aboveskip=0.5cm,
framextopmargin=3pt,
framexbottommargin=3pt,
framexleftmargin=0.4cm,
framesep=0pt,
rulesep=.4pt,
backgroundcolor=\color{gray97},
rulesepcolor=\color{black},
stringstyle=\ttfamily,
showstringspaces=false,
basicstyle=\ttfamily,
commentstyle=\color{green},
keywordstyle=\color{red},
numbers=left,
numbersep=15pt,
numberstyle=\tiny,
numberfirstline=false,
breaklines=true,
 tabsize=2,
 extendedchars=true,
keepspaces,
}
}

\newcommand\FortranStyleA
{
\lstset{
frame=Ltb,
framerule=0pt,
columns=fullflexible,
aboveskip=0.5cm,
framextopmargin=3pt,
framexbottommargin=3pt,
framexleftmargin=0.4cm,
framesep=0pt,
rulesep=.4pt,
backgroundcolor=\color{gray97},
rulesepcolor=\color{black},
stringstyle=\ttfamily,
showstringspaces=false,
basicstyle=\ttfamily,
commentstyle=\color{green},
keywordstyle=\color{red},
numbersep=15pt,
numberstyle=\tiny,
numberfirstline=false,
breaklines=true,
 tabsize=2,
 extendedchars=true,
keepspaces,
}
}

\newcommand\tab[1][1cm]{\hspace*{#1}}
\newcommand{\light}[1]{\textcolor{lightgray}{#1}}
    
\def\signed #1{{\leavevmode\unskip\nobreak\hfil\penalty50\hskip2em
  \hbox{}\nobreak\hfil(#1)%
  \parfillskip=0pt \finalhyphendemerits=0 \endgraf}}

\newsavebox\mybox
\newenvironment{aquote}[1]
  {\savebox\mybox{#1}\begin{quote}}
  {\signed{\usebox\mybox}\end{quote}}


% items enclosed in square brackets are optional; explanation below
\title{Polymorphism}
\author{Jules Kouatchou}
\institute{
  NASA GSFC Code 606 (ASTG)\\
  Greenbelt, Maryland 20771\\[1ex]
  \texttt{Jules.Kouatchou@nasa.gov}
}
\date{October 25, 2018}

\begin{document}

% --- Title page ---
\begin{frame}[plain]
  \titlepage
\end{frame}

\logo{%
  \includegraphics[width=1cm,height=1cm,keepaspectratio]{../../shared/nasa-ball.png}%
  \hspace{\dimexpr\paperwidth-2cm-5pt}%
  \includegraphics[width=1cm,height=1cm,keepaspectratio]{../../shared/ssai-logo.png}%
}


% --- Slide

\begin{frame}{Agenda}

\textcolor{red}{Polymorphism}
    \begin{itemize}
    \item Introduction
    \item Procedure Polymorphism
    \item Data Polymorphism
      \begin{itemize}
      \item Polymorphic Pointer Variable
      \item Allocatable Polymorphic Variable
      \end{itemize}
    \end{itemize}
  

\end{frame}


\frame[containsverbatim]{
  \frametitle{Introduction}
  \begin{itemize}
   \item Use to describe a variety of techniques employed
     by programmers to create flexible and reusable software
     components.
   \item A polymorphic object is an entity, such as a \textbf{variable} or a
     \textbf{procedure},  that can hold or operate on values of differing types during the program's execution.
   \end{itemize}
}

\frame[containsverbatim]{
  \frametitle{Introduction - Cont}
  There are two basic types of polymorphism:
  \begin{description}
   \item[Procedure polymorphism:] Deals with procedures that can operate on a variety of data types and values.
   \item[Data polymorphism:] Deals with program variables that can store and operate on a variety of data types and values.
   \end{description}
}

\frame[containsverbatim]{
  \frametitle{Procedure Polymorphism}
  \begin{itemize}
   \item Occurs when a procedure, such as a function or a subroutine, can take a variety of data types as arguments
   \item Accomplished when a procedure has one or more dummy arguments declared with the \textbf{CLASS} keyword.
   \end{itemize}
 }

 \frame[containsverbatim]{
   \frametitle{Example}

 {\footnotesize
  \begin{semiverbatim}
  \begin{lstlisting}
type, extends(shape) :: polygon
   integer :: color
contains
   procedure :: get_area
   procedure :: set_color
end type polygon

subroutine set_color(plg, color)
    class(polygon), intent(inOut) :: plg
    integer, intent(in) :: color
    plg%color = color
end subroutine set_color
  \end{lstlisting}
\end{semiverbatim}
}
}

\frame[containsverbatim]{
  \frametitle{Comments}
  \begin{itemize}
   \item The \texttt{set\_color} subroutine takes two arguments,
     \texttt{plg} and \texttt{color}.
     \item The \texttt{plg} dummy argument is polymorphic, based on
       the usage of \texttt{class(plygon)}.
     \item The subroutine can operate on objects that satisfy the "is
       a" polygon relationship. So, \texttt{set\_color}  can be called
       with a  \texttt{polygon}, \texttt{circle},
       \texttt{rectangle}, \texttt{square}, or any future type extension of shape.
   \end{itemize}
}


 \frame[containsverbatim]{
   \frametitle{Another Example}

 {\tiny
  \begin{semiverbatim}
  \begin{lstlisting}
subroutine initialize_polygon(plg, color, radius, length, width)
    class(polygon) :: plg
    integer :: color
    real, optional :: radius
    real, optional :: length, width

    plg%color = color
    
    SELECT TYPE (plg)
    type is (polygon)
    class is (circle)
        if (present(radius))  then
           plg%radius = radius
        else
           plg%radius = 0
        endif
    class is (rectangle)
        if (present(length))  then
           plg%length = length
        else
           plg%length = 0
        endif
        if (present(width)) then
            plg%width = width
        else
            plg%width = 0
        endif
    class default
         stop 'initialize: unexpected type for plg object!'
    end select 
end subroutine initialize_polygon
  \end{lstlisting}
\end{semiverbatim}
}
}


\frame[containsverbatim]{
  \frametitle{Data Polymorphism}
 \begin{itemize}
  \item A polymorphic variable is a variable whose data type is
    dynamic at runtime.
  \item It must be a pointer variable, allocatable variable, or a
    dummy argument.
  \end{itemize}
}


\frame[containsverbatim]{
  \frametitle{Example}

  {\footnotesize
  \begin{semiverbatim}
  \begin{lstlisting}
subroutine init(plg)
  class(plygon) :: plg              ! polymorphic dummy argument
  class(plygon), pointer :: p       ! polymorphic pointer variable
  class(plygon), allocatable :: alp ! polymorphic allocatable variable
end subroutine init
  \end{lstlisting}
\end{semiverbatim}
}

\begin{itemize}
\item The \texttt{plg}, \texttt{p} and \texttt{alp} polymorphic variables
  can each hold values of type shape or any type extension of polygon.
\item The \texttt{plg} dummy argument receives its type and value from the actual argument to \texttt{plg} of subroutine \texttt{init()}.
\end{itemize}
}

% --- Slide ---
\frame[containsverbatim]{
  \frametitle{Polymorphic Pointer Variable}
The polymorphic pointer variable \texttt{p} can point to an object of type \texttt{polygon} or any of its extensions.
  {\tiny
  \begin{semiverbatim}
  \begin{lstlisting}
subroutine init(plg)
  class(polygon), target  :: plg
  class(polygon), pointer :: p
  
  select type (plg)
  type is (polygon)
    p => plg
    :  ! shape specific code here
  type is (circle)
    p => plg
    :  ! rectangle specific code here
  type is (rectangle)
    p => plg
    :  ! rectangle specific code here
  type is (square)
    p => plg
    :  ! square specific code here
  class default
    p => null()
  end select
end subroutine init
  \end{lstlisting}
\end{semiverbatim}
}
}

\frame[containsverbatim]{
  \frametitle{Allocatable Polymorphic Variable}
  \begin{itemize}
   \item An allocatable polymorphic variable receives its type and
     optionally its value at the point of its allocation.
   \item By default, the dynamic type of a polymorphic allocatable variable is the same as its declared type after executing an allocate statement.
  \end{itemize}

  {\footnotesize
  \begin{semiverbatim}
  \begin{lstlisting}
class(polygon), allocatable :: alp1, alp2
allocate(alp1)
allocate(rectangle::alp2)
  \end{lstlisting}
\end{semiverbatim}
}
}

\frame[containsverbatim]{
  \frametitle{Example}
  {\tiny
  \begin{semiverbatim}
  \begin{lstlisting}
subroutine init(plg)
  class(polygon) :: plg
  class(polygon), allocatable :: alp
  
  select type (plg)
  type is (polygon)
    allocate(polygon::alp)
    select type(alp)
    type is (polygon)
      alp = plg   ! copy sh
    end select
  type is (circle)
    allocate(circle::alp)
    select type(alp)
    type is (circle)
      alp = plg   ! copy sh
    end select
  type is (rectangle)
    allocate(rectangle::alp)
     select type (alp)
     type is (rectangle)
       alp = plg ! copy plg
     end select
  type is (square)
    allocate(square::alp)
    select type (alp)
    type is (square)
      alp = plg   ! copy plg
    end select
  end select
end subroutine init
  \end{lstlisting}
\end{semiverbatim}
}
}



\frame[containsverbatim]{
  \frametitle{Exercise }
  \begin{itemize}
  \item For each child class, implement a method that computes the perimeter of a polygon
    \item Write a subroutine that takes as argument an arbitrary \texttt{polygon}
      and prints the perimeter of the \texttt{polygon}.
    \item Write a simple program that initialzes various polygons and calls the above subroutine.
   \end{itemize}
}



\end{document}
