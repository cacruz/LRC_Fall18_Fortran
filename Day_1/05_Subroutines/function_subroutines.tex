\documentclass[11pt]{beamer}
\setbeamertemplate{navigation symbols}{}
 \setbeamercovered{transparent}
\usepackage{listings}
%\usetheme{Copenhagen}
\usetheme{Singapore}
%\usetheme{Madrid}
%\usetheme{Hannover}
%\usetheme{boxes}
%\usetheme{Boadilla}
\usefonttheme[onlymath]{serif}
\usecolortheme{beaver}
\usepackage{textpos}
\usepackage{fancyvrb}
\usepackage{xcolor}
\usepackage{multicol}
\usepackage{lipsum}
\parskip 1ex

\newcommand\FontAcolumn{\fontsize{6}{7.2}\selectfont}
\newcommand\FontBcolumn{\fontsize{8}{7.2}\selectfont}
\newcommand\FontCcolumn{\fontsize{10}{7.2}\selectfont}
\newcommand\FontDcolumn{\fontsize{11}{7.2}\selectfont}

\definecolor{gray97}{gray}{.97}
\definecolor{gray75}{gray}{.75}
\definecolor{gray75}{gray}{.45}

%%%%%%%%%%%%%%%%%%%%%%%%%%%%%%%%%%%%%%%%%%%%%%%%%%%%%%%%
%%%% This is used to add line number to source code %%%%
%%%%%%%%%%%%%%%%%%%%%%%%%%%%%%%%%%%%%%%%%%%%%%%%%%%%%%%%
\lstset{
        language=fortran,        % language of the code
        basicstyle=\small\ttfamily, % size of the fonts used for the code
        %basicstyle=\normalsize, % size of the fonts used for the code
        numbers=left,           % where to put line numbers
        numberstyle=\tiny,     % size of the fonts used for line numbers
        stepnumber=1, 
        numbersep=8pt,
        numberfirstline=false,
        showstringspaces=false, % underline spaces within strings
        aboveskip=-20pt,
        numbersep=15pt,          % how far the line-numbers are from the code
        frame=Ltb,         % addition of a left frame line on source
        % code
        framerule=0pt,
        columns=fullflexible,
        framextopmargin=3pt,
        framexbottommargin=3pt,
        framexleftmargin=0.4cm,
        framesep=0pt,
        rulesep=.4pt,
        backgroundcolor=\color{gray97},
        rulesepcolor=\color{black},
        stringstyle=\ttfamily,
        showstringspaces=false,
        basicstyle=\ttfamily,
        keywordstyle=\color{red}, % color keywords
        commentstyle=\color{green}    % color comments
        }
\lstdefinestyle{Fortran}{language=[90]Fortran}

\newcommand\FortranStyle
{
\lstset{
frame=Ltb,
framerule=0pt,
columns=fullflexible,
aboveskip=0.5cm,
framextopmargin=3pt,
framexbottommargin=3pt,
framexleftmargin=0.4cm,
framesep=0pt,
rulesep=.4pt,
backgroundcolor=\color{gray97},
rulesepcolor=\color{black},
stringstyle=\ttfamily,
showstringspaces=false,
basicstyle=\ttfamily,
commentstyle=\color{green},
keywordstyle=\color{red},
numbers=left,
numbersep=15pt,
numberstyle=\tiny,
numberfirstline=false,
breaklines=true,
 tabsize=2,
 extendedchars=true,
keepspaces,
}
}

\newcommand\FortranStyleA
{
\lstset{
frame=Ltb,
framerule=0pt,
columns=fullflexible,
aboveskip=0.5cm,
framextopmargin=3pt,
framexbottommargin=3pt,
framexleftmargin=0.4cm,
framesep=0pt,
rulesep=.4pt,
backgroundcolor=\color{gray97},
rulesepcolor=\color{black},
stringstyle=\ttfamily,
showstringspaces=false,
basicstyle=\ttfamily,
commentstyle=\color{green},
keywordstyle=\color{red},
numbersep=15pt,
numberstyle=\tiny,
numberfirstline=false,
breaklines=true,
 tabsize=2,
 extendedchars=true,
keepspaces,
}
}

\newcommand\tab[1][1cm]{\hspace*{#1}}
\newcommand{\light}[1]{\textcolor{lightgray}{#1}}
    
\def\signed #1{{\leavevmode\unskip\nobreak\hfil\penalty50\hskip2em
  \hbox{}\nobreak\hfil(#1)%
  \parfillskip=0pt \finalhyphendemerits=0 \endgraf}}

\newsavebox\mybox
\newenvironment{aquote}[1]
  {\savebox\mybox{#1}\begin{quote}}
  {\signed{\usebox\mybox}\end{quote}}

  
% items enclosed in square brackets are optional; explanation below
\title{Functions and Subroutines}
%\subtitle{Introduction}
\author{Carlos Cruz\\
Jules Kouatchou\\
Bruce Van Aartsen}
\institute{
  NASA GSFC Code 606 (ASTG)\\
  Greenbelt, Maryland 20771\\[1ex]
%  \texttt{carlos.a.cruz@nasa.gov}
}
\date{October 24, 2018}

\begin{document}

% --- Title page ---
\begin{frame}[plain]
  \titlepage
\end{frame}

\logo{%
  \includegraphics[width=1cm,height=1cm,keepaspectratio]{../../shared/nasa-ball.png}%
  \hspace{\dimexpr\paperwidth-2cm-5pt}%
  \includegraphics[width=1cm,height=1cm,keepaspectratio]{../../shared/ssai-logo.png}%
}

% --- Slide

\frame[containsverbatim]{
  \frametitle{Rationale}
  \begin{itemize}
    \item Actions/operations which have to be performed more than once
    \item One big problem split into clearer and simpler sub-problems
  \end{itemize}

 }

\frame[containsverbatim]{
  \frametitle{Functions}
  \begin{itemize}
    \item Special type of Fortran subprogram that is expected to
      return a single result or answer
    \item Typically accept some kind of input information and based on that information, return a result.
  \end{itemize}

 }

 \frame[containsverbatim]{
  \frametitle{Syntax for a User-Defined Function }
  \begin{semiverbatim}
  \begin{lstlisting}
 function <name> ( <arguments> ) result(returnedValue)
   <declarations>
   <type> :: returnedValue
   <body of function>
   returnedValue = expression
 end function <name>
  \end{lstlisting}
\end{semiverbatim}
The $<type>$ is one of the Fortran data types: real, integer, logical, character, or complex.
}

 \frame[containsverbatim]{
  \frametitle{Example of Function: Leap Year }
  \begin{semiverbatim}
  \begin{lstlisting}
      FUNCTION calcAverage (a, b, c) result(av)
       implicit none
       real(kind=4), intent(in) :: a, b, c
       real(kind=4)             :: av

       av = (a + b + c)/3.0

      END FUNCTION calcAverage
  \end{lstlisting}
\end{semiverbatim}

}

 \frame[containsverbatim]{
   \frametitle{Few Fortran Intrinsic Functions}

\begin{verbatim}
abs      int      trim         aimag     maxval
exp      nint     len          real      minval
log      real     len_trim     cmplx     sum
exp      mod
cos
sin
tan
acos
asin
atan
\end{verbatim}
   }




\frame[containsverbatim]{
  \frametitle{Subroutines}
  \begin{itemize}
    \item Accept some kind of input information and based on that
      information, return a result or series of results.
    \item Each of the passed arguments must be declared and the
      declaration must include the type and the intent.
    \item The arguments in the calling routine and the subroutine must match and are matched up by position.
  \end{itemize}

 }

 \frame[containsverbatim]{
  \frametitle{Syntax for a Subroutine }
  \begin{semiverbatim}
  \begin{lstlisting}
 subroutine <name> ( <arguments> )
    <declarations>
    <body of subroutine>
    return
 end subroutine <name>
  \end{lstlisting}
\end{semiverbatim}
}

\frame[containsverbatim]{
  \frametitle{Example of Subroutine: Compute Sum/Average of Numbers }
  \begin{semiverbatim}
  \begin{lstlisting}
       subroutine sumAverage (a, b, c, sm, av)
       real(kind=4), intent(in) :: a, b, c
       real(kind=4), intent(out) :: sm, av

       sm = a + b + c
       av = sm / 3.0

       return
       end subroutine sumAverage
  \end{lstlisting}
\end{semiverbatim}
}


\frame[containsverbatim]{
  \frametitle{Exercise }
Write a subroutine that returns the sum, the average and the standard deviation
of 5 numbers.
\begin{eqnarray}
s & = & \sum_{i=1}^{5}{a_i} \nonumber \\
\mu & =& \frac{s}{5} \nonumber \\
std & = & \frac{1}{5} \sum_{i=1}^{5}{(a_i}-\mu)^2}  \nonumber
\end{eqnarray}

}


\end{document}

