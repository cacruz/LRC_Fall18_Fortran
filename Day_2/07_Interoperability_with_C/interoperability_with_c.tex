\documentclass[11pt]{beamer}
\setbeamertemplate{navigation symbols}{}
 \setbeamercovered{transparent}
\usepackage{listings}
%\usetheme{Copenhagen}
\usetheme{Singapore}
%\usetheme{Madrid}
%\usetheme{Hannover}
%\usetheme{boxes}
%\usetheme{Boadilla}
\usefonttheme[onlymath]{serif}
\usecolortheme{beaver}
\usepackage{textpos}
\usepackage{fancyvrb}
\usepackage{xcolor}
\usepackage{multicol}
\usepackage{lipsum}
\parskip 1ex

\newcommand\FontAcolumn{\fontsize{6}{7.2}\selectfont}
\newcommand\FontBcolumn{\fontsize{8}{7.2}\selectfont}
\newcommand\FontCcolumn{\fontsize{10}{7.2}\selectfont}
\newcommand\FontDcolumn{\fontsize{11}{7.2}\selectfont}

\definecolor{gray97}{gray}{.97}
\definecolor{gray75}{gray}{.75}
\definecolor{gray75}{gray}{.45}

%%%%%%%%%%%%%%%%%%%%%%%%%%%%%%%%%%%%%%%%%%%%%%%%%%%%%%%%
%%%% This is used to add line number to source code %%%%
%%%%%%%%%%%%%%%%%%%%%%%%%%%%%%%%%%%%%%%%%%%%%%%%%%%%%%%%
\lstset{
        language=fortran,        % language of the code
        basicstyle=\small\ttfamily, % size of the fonts used for the code
        %basicstyle=\normalsize, % size of the fonts used for the code
        numbers=left,           % where to put line numbers
        numberstyle=\tiny,     % size of the fonts used for line numbers
        stepnumber=1, 
        numbersep=8pt,
        numberfirstline=false,
        showstringspaces=false, % underline spaces within strings
        aboveskip=-20pt,
        numbersep=15pt,          % how far the line-numbers are from the code
        frame=Ltb,         % addition of a left frame line on source
        % code
        framerule=0pt,
        columns=fullflexible,
        framextopmargin=3pt,
        framexbottommargin=3pt,
        framexleftmargin=0.4cm,
        framesep=0pt,
        rulesep=.4pt,
        backgroundcolor=\color{gray97},
        rulesepcolor=\color{black},
        stringstyle=\ttfamily,
        showstringspaces=false,
        basicstyle=\ttfamily,
        keywordstyle=\color{red}, % color keywords
        commentstyle=\color{green}    % color comments
        }


\lstdefinestyle{Fortran}{language=[90]Fortran}

\newcommand\FortranStyle
{
\lstset{
frame=Ltb,
framerule=0pt,
columns=fullflexible,
aboveskip=0.5cm,
framextopmargin=3pt,
framexbottommargin=3pt,
framexleftmargin=0.4cm,
framesep=0pt,
rulesep=.4pt,
backgroundcolor=\color{gray97},
rulesepcolor=\color{black},
stringstyle=\ttfamily,
showstringspaces=false,
basicstyle=\ttfamily,
commentstyle=\color{green},
keywordstyle=\color{red},
numbers=left,
numbersep=15pt,
numberstyle=\tiny,
numberfirstline=false,
breaklines=true,
 tabsize=2,
 extendedchars=true,
keepspaces,
}
}

\newcommand\FortranStyleA
{
\lstset{
frame=Ltb,
framerule=0pt,
columns=fullflexible,
aboveskip=0.5cm,
framextopmargin=3pt,
framexbottommargin=3pt,
framexleftmargin=0.4cm,
framesep=0pt,
rulesep=.4pt,
backgroundcolor=\color{gray97},
rulesepcolor=\color{black},
stringstyle=\ttfamily,
showstringspaces=false,
basicstyle=\ttfamily,
commentstyle=\color{green},
keywordstyle=\color{red},
numbersep=15pt,
numberstyle=\tiny,
numberfirstline=false,
breaklines=true,
 tabsize=2,
 extendedchars=true,
keepspaces,
}
}

\newcommand\tab[1][1cm]{\hspace*{#1}}
\newcommand{\light}[1]{\textcolor{lightgray}{#1}}
    
\def\signed #1{{\leavevmode\unskip\nobreak\hfil\penalty50\hskip2em
  \hbox{}\nobreak\hfil(#1)%
  \parfillskip=0pt \finalhyphendemerits=0 \endgraf}}

\newsavebox\mybox
\newenvironment{aquote}[1]
  {\savebox\mybox{#1}\begin{quote}}
  {\signed{\usebox\mybox}\end{quote}}


% items enclosed in square brackets are optional; explanation below
\title{Interoperability with C}
\author{Jules Kouatchou}
\institute{
  NASA GSFC Code 606 (ASTG)\\
  Greenbelt, Maryland 20771\\[1ex]
  \texttt{Jules.Kouatchou@nasa.gov}
}
\date{October 25, 2018}

\begin{document}

% --- Title page ---
\begin{frame}[plain]
  \titlepage
\end{frame}

\logo{%
  \includegraphics[width=1cm,height=1cm,keepaspectratio]{../../shared/nasa-ball.png}%
  \hspace{\dimexpr\paperwidth-2cm-5pt}%
  \includegraphics[width=1cm,height=1cm,keepaspectratio]{../../shared/ssai-logo.png}%
}


% --- Slide

\begin{frame}{Agenda}

\textcolor{red}{Interoperability with C}
    \begin{itemize}
    \item ISO\_C\_BINDING
    \item Intrinsic Types
    \item Interoperable procedures
    \item Interoperable data
    \end{itemize}
  

\end{frame}


\frame[containsverbatim]{
  \frametitle{Introduction }
  \begin{itemize}
   \item Fotran provides standards for accessing libraries and
     procedures developed in C
    \item Fotran also provides standards for accessing Fortran libraries and
      procedures developed from  C
    \item Interoperability enforced by requirements on Fortran syntax
    \item Use of these features requires some familiarity with both C
      and Fortran
   \end{itemize}
 }

 \frame[containsverbatim]{
  \frametitle{ISO\_C\_BINDING}
Intrinsic module that provides:
  \begin{itemize}
   \item Named constants for declaring Fortran data which interoperates with C data
    \item Small number of procedures for managing pointers and addresses
   \end{itemize}
 
 To avoid naming conflicts it is recommended that the ONLY options is
 used in the USE statement.
}


\frame[containsverbatim]{
  \frametitle{Intrinsic Data Types}
For each C data type provided by the vendor there is an equivalent named constant in
ISO\_C\_BINDING
  \begin{itemize}
   \item Value of the named constant specifies the \textbf{KIND} for the corresponding Fortran data type
    \item Support for: INTEGER, REAL, COMPLEX, LOGICAL, and CHARACTER types
   \end{itemize}
}

\frame[containsverbatim]{
  \frametitle{Intrinsic Types}

\begin{center}
\begin{tabular}{|l|l|l|}
\hline \hline
\textbf{Fortran Type} & \textbf{Named constant from} & \textbf{C Type} \\
     & \textbf{C type ISO\_C\_BINDING} &     \\  \hline \hline
INTEGER & C\_INT & int \\ \hline
INTEGER & C\_LONG & short int  \\ \hline
INTEGER & C\_INT32 & int32\_t  \\ \hline
  INTEGER & C\_INT64 & int64\_t  \\ \hline
REAL & C\_FLOAT & float  \\ \hline
REAL & C\_DOUBLE & double  \\ \hline
COMPLEX & C\_FLOAT\_COMPLEX & float\_Complex  \\ \hline
LOGICAL & C\_BOOL & \_Bool  \\ \hline
CHARACTER & C\_CHAR \& char \\ \hline \hline
\end{tabular} 
\end{center}
}

\frame[containsverbatim]{
  \frametitle{Example of Declarations }
  Example of Fortran declaration interoperable with C double
  \begin{semiverbatim}
  \begin{lstlisting}
   use, intrinsic :: ISO_C_Binding, only: C_int, C_char, C_double
   IMPLICIT NONE

   integer, parameter :: maxLengthChar= 100

   character(kind=C_CHAR), DIMENSION(*) :: stationName
   real(KIND=C_FLOAT) :: latitude
   real(KIND=C_FLOAT) :: longitude
\end{lstlisting}
\end{semiverbatim}
}


\frame[containsverbatim]{
  \frametitle{Intrinsic Procedures}
  \begin{itemize}
   \item \textbf{C\_LOC} (var) $\rightarrow$ Returns C address (type C\_PTR) of var
   \item \textbf{C\_FUNLOC} (proc) $\rightarrow$ Returns C address (type C\_FUNPTR) of procedure
   \item \textbf{C\_ASSOCIATED} (cPtr1 [, cPtr2]) $\rightarrow$
     Returns false if cPtr1 is a null C pointer or if cPtr2 is present
     with a different value
   \item \textbf{C\_F\_POINTER} (cPtr1, fPtr [, shape]) $\rightarrow$
     Associates Fortran pointer, fPtr1, with address cPtr1 (type
     C\_PTR).  Shape is required when fPtr1 is an array pointer.
    \item \textbf{C\_F\_PROCPOINTER} (cPtr1, fPtr)  $\rightarrow$  Associates Fortran procedure pointer, fPtr1 with the address of
interoperable C procedure cPtr1 (type C\_FUNPTR)
   \end{itemize}
 }

 \frame[containsverbatim]{
  \frametitle{Interoperable Procedures}
  \begin{itemize}
  \item A Fortran procedure is interoperable if
    \begin{itemize}
     \item It has an explicit interface
     \item It has been declared with the BIND attribute
      \item The number of dummy arguments is equal to the number of formal
parameters in the prototype and are in the same relative positions as the
C parameter list
      \item All dummy arguments are interoperable
    \end{itemize}
  \item Return values
    \begin{itemize}
    \item An interoperable Fortran function must have a result that is scalar and
      interoperable
    \item For a subroutine, the C prototype must have a void result
    \end{itemize}
  \item Caveats
    \begin{itemize}
  \item Interoperable functions cannot return array values
   \item Fortran procedures cannot interoperate with C functions that take a
variable number of arguments (the C language specification allows this)
    \end{itemize}
      \end{itemize}
 }

\frame[containsverbatim]{
  \frametitle{Example of Interoperable Fortran Procedure Interface }
  {\small
\begin{semiverbatim}
  \begin{lstlisting}
INTERFACE
   FUNCTION func (i, j, k, l, m), BIND (C, name=''C_Func'')
       USE, INTRINSIC :: ISO_C_BINDING
       INTEGER (C_SHORT) :: func
       INTEGER (C_INT), VALUE :: i
       REAL (C_DOUBLE) :: j
       INTEGER (C_INT) :: k, l(10)
       TYPE (C_PTR), VALUE :: m
   END FUNCTION func
END INTERFACE
…
short C_Func (int i, double *j, int *k, int l[10], void *m)
\end{lstlisting}
\end{semiverbatim}
}
}

\frame[containsverbatim]{
  \frametitle{Binding Labels for Procedures}
  A \textbf{binding label} is a value that specifies the name by
which a procedure with the BIND attribute is known
  \begin{itemize}
  \item Has global scope
    \item By default, it is the lower-case version of the Fortran name
   \end{itemize}
 }

 \frame[containsverbatim]{
  \frametitle{Interoperable Data}
  Fortran data is interoperable if an equivalent
data declaration can be made in C and the
data is said to be interoperable
  \begin{itemize}
  \item Scalar and array variables are interoperable
    \item Dynamic arrays can be passed between the two
      languages
    \item The BIND attribute is required for a Fortran derived
      type to be interoperable
    \item C variables with external linkage can interoperate
with Fortran common blocks or module variables
that have the BIND attribute
   \end{itemize}
 }

 \frame[containsverbatim]{
  \frametitle{Interoperability of Variables}
  \begin{itemize}
  \item Fortran scalars are interoperable if
    \begin{itemize}
    \item the type and type parameters are interoperable with a scalar
      C variable
      \item they are not declared as pointers nor have the allocatable attribute
     \end{itemize}
   \item Fortran arrays are interoperable if
        \begin{itemize}
    \item The type and type parameters are interoperable
      \item they are of explicit shape or assumed size
      \end{itemize}
      \item Fortran arrays interoperate with C arrays of the same type, type parameters and shape,
but with reversed subscripts
   \end{itemize}
 }

 \frame[containsverbatim]{
  \frametitle{Example of Interoperability of Variables}
  \begin{semiverbatim}
  \begin{lstlisting}
real :: A(3,4)
real :: A(3,*)
real :: A(:,:)   ! not allowed

INTEGER :: A(18, 3:7, *)
…
int b[] [5] [18]
\end{lstlisting}
\end{semiverbatim}
}


 \frame[containsverbatim]{
  \frametitle{Derived Types}
  \begin{itemize}
  \item Interoperable Fortran derived types must
    \begin{itemize}
    \item Specify the BIND (C) attribute
    \item Have the same number of components as the C struct type
     \item Have components with type and type parameters that are
interoperable with the types of the corresponding components of the
C struct type
     \end{itemize}
   \item Components of the Fortran derived type
        \begin{itemize}
    \item Correspond to the C struct type components declared in the same
relative position
      \item Corresponding components do not need to have the same name
      \end{itemize}
   \end{itemize}
 }

  \frame[containsverbatim]{
  \frametitle{Example of Derived Type}
  \begin{semiverbatim}
  \begin{lstlisting}
TYPE, BIND (C) :: fType
    INTEGER (C_INT) :: i, j
    REAL (C_FLOAT) :: s
END TYPE fType
…
typedef struct {
    int m, n;
    float r;
} cType
\end{lstlisting}
\end{semiverbatim}
}


\frame[containsverbatim]{
  \frametitle{Global Data}
  \begin{itemize}
 \item A C variable with external linkage can interoperate
with a Fortran common block or variable that has the
BIND attribute
\item C variable with external linkage interoperates with a
common block specified in a BIND statement in one of
two ways:
  \begin{itemize}
   \item The C variable is a struct type and the elements are
interoperable with the members of the common block
   \item Or the common block contains only one interoperable
     variable
   \end{itemize}
\item Only one variable can be associated with a C variable
with external linkage
   \end{itemize}
 }

   \frame[containsverbatim]{
  \frametitle{Example of Global Data}
  \begin{semiverbatim}
  \begin{lstlisting}
use ISO_C_BINDING
COMMON /COM/ r, s
REAL(C_FLOAT) :: r, s
BIND(C) :: /COM/
…
struct {
   float r, s;
} com; /* external */

void setter() {
   com.r = 3;
   com.s = 4;
}
\end{lstlisting}
\end{semiverbatim}
}

\frame[containsverbatim]{
  \frametitle{Array Variables}
  \begin{itemize}
\item A Fortran array of rank one is not interoperable with a
multidimensional C array
\item Polymorphic, allocatable, and pointer arrays are never
interoperable
\item A Fortran array of type character with a kind type of
C\_CHAR is interoperable with a C string (C null
character as last element of the array)
\begin{itemize}
\item ISO\_C\_BINDING provides the constant C\_NULL\_CHAR
     \end{itemize}
   \end{itemize}
 }


 \frame[containsverbatim]{
  \frametitle{Dynamic Arrays}
  \begin{itemize}
\item  C pointers are the mechanism for passing dynamic
  arrays between the two languages
    \begin{itemize}
\item an allocated allocatable Fortran array can be passed to C
\item an array allocated in C can be passed to a Fortran pointer
\item a Fortran pointer target or assumed-shape array (no bounds
specified) cannot be passed to C
   \end{itemize}
\item ISO\_C\_BINDING provides
  \begin{itemize}
    \item C\_PTR is the derived type for interoperating with any C
      object pointer type
    \item C\_NULL\_PTR is the named constant of type C\_PTR with the value NULL in C
   \end{itemize}
   \end{itemize}
 }


    \frame[containsverbatim]{
  \frametitle{Example of Interoperable Dynamic Arrays - C}
{\tiny
\begin{semiverbatim}
  \begin{lstlisting}
typedef struct {
   int lenc, lenf;
   float *c, *f;
} pass;

int main () {
…
pass *arrays=(pass*)malloc(sizeof(pass));
(*arrays).lenc = 2;
arrays->c =malloc((*arrays).lenc*sizeof(float));
a[0] = 10.0;
a[1] = 20.0;

for(i=0;i<(*arrays).lenc;i++) {
   *(arrays->c+i)=a[i];
}
/* Calling Fortran routine "simulation" */
simulation(arrays);
\end{lstlisting}
\end{semiverbatim}
}
}

   \frame[containsverbatim]{
  \frametitle{Example of Interoperable Dynamic Arrays - Fortran}
  \begin{semiverbatim}
  \begin{lstlisting}
SUBROUTINE simulation(arrays) bind(c)
…

TYPE, BIND(c) :: pass
   integer (C_INT) :: lenc, lenf
   TYPE (C_PTR) :: c, f
END TYPE pass

TYPE (pass), INTENT(INOUT) :: arrays
REAL (C_FLOAT), POINTER : cArray (:)
CALL C_F_POINTER(arrays%c,cArray, (/arrays%lenc/))
print*, cArray

\end{lstlisting}
\end{semiverbatim}
}
 
\frame[containsverbatim]{
  \frametitle{Exercise}
  \begin{itemize}
   \item Write a C subroutine that computes the averrage and standard
     deviation of an array with $n$ entries.
   \item Call the subroutine within Fortran.
   \end{itemize}
 }

\end{document}
