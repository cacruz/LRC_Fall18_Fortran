\documentclass[11pt]{beamer}
\setbeamertemplate{navigation symbols}{}
 \setbeamercovered{transparent}
\usepackage{listings}
%\usetheme{Copenhagen}
\usetheme{Singapore}
%\usetheme{Madrid}
%\usetheme{Hannover}
%\usetheme{boxes}
%\usetheme{Boadilla}
\usefonttheme[onlymath]{serif}
\usecolortheme{beaver}
\usepackage{textpos}
\usepackage{fancyvrb}
\usepackage{xcolor}
\usepackage{multicol}
\usepackage{lipsum}
\parskip 1ex

\newcommand\FontAcolumn{\fontsize{6}{7.2}\selectfont}
\newcommand\FontBcolumn{\fontsize{8}{7.2}\selectfont}
\newcommand\FontCcolumn{\fontsize{10}{7.2}\selectfont}
\newcommand\FontDcolumn{\fontsize{11}{7.2}\selectfont}

\definecolor{gray97}{gray}{.97}
\definecolor{gray75}{gray}{.75}
\definecolor{gray75}{gray}{.45}

%%%%%%%%%%%%%%%%%%%%%%%%%%%%%%%%%%%%%%%%%%%%%%%%%%%%%%%%
%%%% This is used to add line number to source code %%%%
%%%%%%%%%%%%%%%%%%%%%%%%%%%%%%%%%%%%%%%%%%%%%%%%%%%%%%%%
\lstset{
        language=fortran,        % language of the code
        basicstyle=\small\ttfamily, % size of the fonts used for the code
        %basicstyle=\normalsize, % size of the fonts used for the code
        numbers=left,           % where to put line numbers
        numberstyle=\tiny,     % size of the fonts used for line numbers
        stepnumber=1, 
        numbersep=8pt,
        numberfirstline=false,
        showstringspaces=false, % underline spaces within strings
        aboveskip=-20pt,
        numbersep=15pt,          % how far the line-numbers are from the code
        frame=Ltb,         % addition of a left frame line on source
        % code
        framerule=0pt,
        columns=fullflexible,
        framextopmargin=3pt,
        framexbottommargin=3pt,
        framexleftmargin=0.4cm,
        framesep=0pt,
        rulesep=.4pt,
        backgroundcolor=\color{gray97},
        rulesepcolor=\color{black},
        stringstyle=\ttfamily,
        showstringspaces=false,
        basicstyle=\ttfamily,
        keywordstyle=\color{red}, % color keywords
        commentstyle=\color{green}    % color comments
        }

\lstdefinestyle{Fortran}{language=[90]Fortran}

\newcommand\FortranStyle
{
\lstset{
frame=Ltb,
framerule=0pt,
columns=fullflexible,
aboveskip=0.5cm,
framextopmargin=3pt,
framexbottommargin=3pt,
framexleftmargin=0.4cm,
framesep=0pt,
rulesep=.4pt,
backgroundcolor=\color{gray97},
rulesepcolor=\color{black},
stringstyle=\ttfamily,
showstringspaces=false,
basicstyle=\ttfamily,
commentstyle=\color{green},
keywordstyle=\color{red},
numbers=left,
numbersep=15pt,
numberstyle=\tiny,
numberfirstline=false,
breaklines=true,
 tabsize=2,
 extendedchars=true,
keepspaces,
}
}

\newcommand\FortranStyleA
{
\lstset{
frame=Ltb,
framerule=0pt,
columns=fullflexible,
aboveskip=0.5cm,
framextopmargin=3pt,
framexbottommargin=3pt,
framexleftmargin=0.4cm,
framesep=0pt,
rulesep=.4pt,
backgroundcolor=\color{gray97},
rulesepcolor=\color{black},
stringstyle=\ttfamily,
showstringspaces=false,
basicstyle=\ttfamily,
commentstyle=\color{green},
keywordstyle=\color{red},
numbersep=15pt,
numberstyle=\tiny,
numberfirstline=false,
breaklines=true,
 tabsize=2,
 extendedchars=true,
keepspaces,
}
}

\newcommand\tab[1][1cm]{\hspace*{#1}}
\newcommand{\light}[1]{\textcolor{lightgray}{#1}}
    
\def\signed #1{{\leavevmode\unskip\nobreak\hfil\penalty50\hskip2em
  \hbox{}\nobreak\hfil(#1)%
  \parfillskip=0pt \finalhyphendemerits=0 \endgraf}}

\newsavebox\mybox
\newenvironment{aquote}[1]
  {\savebox\mybox{#1}\begin{quote}}
  {\signed{\usebox\mybox}\end{quote}}


% items enclosed in square brackets are optional; explanation below
\title{IO Enhancements}
\author{Jules Kouatchou}
\institute{
  NASA GSFC Code 606 (ASTG)\\
  Greenbelt, Maryland 20771\\[1ex]
  \texttt{Jules.Kouatchou@nasa.gov}
}
\date{October 25, 2018}

\begin{document}

% --- Title page ---
\begin{frame}[plain]
  \titlepage
\end{frame}

\logo{%
  \includegraphics[width=1cm,height=1cm,keepaspectratio]{../../shared/nasa-ball.png}%
  \hspace{\dimexpr\paperwidth-2cm-5pt}%
  \includegraphics[width=1cm,height=1cm,keepaspectratio]{../../shared/ssai-logo.png}%
}


% --- Slide

\begin{frame}{Agenda}

\textcolor{red}{IO Enhancements}
    \begin{itemize}
    \item Stream IO
    \item Asynchronous I/O
    \end{itemize}

%\textcolor{red}{IO Enhancements}
%    \begin{itemize}
%    \item Major Extensions
%    \begin{itemize}
%    \item Stream IO
%    \item Derived Type IO
%    \end{itemize}
%    \item Miscellaneous
%    \begin{itemize}
%    \item Recursive IO
%    \item ISO\_FORTRAN\_ENV
%    \item New Statements
%    \end{itemize}
%    \end{itemize}
  

\end{frame}

\frame[containsverbatim]{
  \frametitle{Steam I/O - 1}
Stream access is a new method for allowing fine-grained, 
random positioning within a file for read/write operations.
\begin{itemize}
\item Complements pre-existing DIRECT and SEQUENTIAL access
\item Advantages:
   \begin{itemize}
   \item Random access (as with DIRECT)
   \item Arbitrary record lengths (as with SEQUENTIAL)
   \end{itemize}
\item Disadvantages:
   \begin{itemize}
   \item Presumably poorer performance than both DIRECT and SEQUENTIAL
   \item Lack of record separators increases risk of inability to read 
         file under small changes.
   \item Index for positioning within file might be less natural than those for DIRECT.
   \end{itemize}
\end{itemize}

}

\frame[containsverbatim]{
  \frametitle{Steam I/O - 2}

\begin{semiverbatim}
\begin{lstlisting}
OPEN(unit, ACCESS = ‘STREAM’) 
! both formatted and unformatted files

READ(unit, POS=n) x,y,z 
! File starts at position POS=1
! Position is specified in “file storage units” -
!     usually bytes

INQUIRE(unit, POS=currentPosition , ...)
! Formatted I/O must use POS obtained from INQUIRE()
!       (or POS=1)
\end{lstlisting}
\end{semiverbatim}
}

\frame[containsverbatim]{
  \frametitle{Examples}
  Check the files:
  \begin{itemize}
  \item \textit{writeUstream.F90}
  \item \textit{readUstream.F90}
  \end{itemize}
 }

\frame[containsverbatim]{
  \frametitle{Asynchronous I/O - 1}
Potential performance enhancement allowing some I/O
operations to be performed
in parallel with other computations.
\begin{itemize}
\item To open a file for asynchronous operations, the new
      optional keyword \textbf{ASYNCHRONOUS} is used.
\item An asynchronous read/write operation is initiated with the
      same keyword.
\item An optional keyword, \textbf{ID}, can be used to return a handle for
      later use in identifying specific pending operations
\end{itemize}
}

\frame[containsverbatim]{   
  \frametitle{Asynchronous I/O - 2}

\begin{semiverbatim}
\begin{lstlisting}
OPEN(10, ..., ASYNCHRONOUS=’yes’)
WRITE(10,..., id=id, ASYNCHRONOUS=’yes’) A
CALL do_something(....) ! Not involving A here 
WAIT(10, id=id) ! Blocks here until A has been written 
CALL do_something(...) ! OK to use A here
\end{lstlisting}
\end{semiverbatim}
}

\frame[containsverbatim]{
  \frametitle{Asynchronous I/O - 3}
If the asynchronous file access is performed in a procedure 
other than the one called for OPEN, the data involved has to be 
declared with ”asynchronous” attribute

{\small
\begin{semiverbatim}
\begin{lstlisting}
OPEN(10, ..., asynchronous=’yes’) 
CALL async_write(10, A, id)
CALL do_something_else_here() 
WAIT(10, id=id)
  ...
SUBROUTINE async_write(iu, data, id)
  INTEGER, INTENT(IN) :: iu
  INTEGER, INTENT(IN), DIMENSION(:), ASYNCHRONOUS :: data 
  INTEGER, INTENT(OUT) :: id
  ...
  WRITE(iu, id=id, asynchronous=’yes’) data
  ...
END SUBROUTINE async_write
\end{lstlisting}
\end{semiverbatim}
}
}

\frame[containsverbatim]{
  \frametitle{Asynchronous I/O - 4}
An alternative for calling \textbf{WAIT} is to periodically 
call \textbf{INQUIRE} to check the status of the operation 
and in the meantime keep on doing something else

{\small
\begin{semiverbatim}
\begin{lstlisting}
LOGICAL :: status
...
OPEN(10, ..., asynchronous=’yes’) 
WRITE(10,..., id=id, asynchronous=’yes’) A 
DO WHILE (!status)
   CALL do_something(....) ! Not involving A
   INQUIRE (10, id=id, pending=status) 
END DO
\end{lstlisting}
\end{semiverbatim}
}
}


\frame[containsverbatim]{
  \frametitle{Exercises}
  Edit the file \textit{exampleAsyncIO.F90} to:
  \begin{itemize}
  \item Write the ''regular version''  (without using asynchronize I/O)
    of the routine \textit{doingAsyncIO}. You can call the new routine \textit{notDoingAsyncIO}.
  \item Time the four calls to \textit{validWrite},
    \textit{invalidWrite}, \textit{doingAsyncIO}, and \textit{notDoingAsyncIO}.
  \end{itemize}
 }

% \frame[containsverbatim]{
%   \frametitle{Introduction }

% }

% \frame[containsverbatim]{
% \frametitle{Introduction }
% \begin{semiverbatim}
% \begin{lstlisting}

% \end{lstlisting}
% \end{semiverbatim}

% }

\end{document}
