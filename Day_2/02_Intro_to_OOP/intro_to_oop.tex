\documentclass[11pt]{beamer}
\setbeamertemplate{navigation symbols}{}
 \setbeamercovered{transparent}
\usepackage{listings}
%\usetheme{Copenhagen}
\usetheme{Singapore}
%\usetheme{Madrid}
%\usetheme{Hannover}
%\usetheme{boxes}
%\usetheme{Boadilla}
\usefonttheme[onlymath]{serif}
\usecolortheme{beaver}
\usepackage{textpos}
\usepackage{fancyvrb}
\usepackage{xcolor}
\usepackage{multicol}
\usepackage{lipsum}
\parskip 1ex

\newcommand\FontAcolumn{\fontsize{6}{7.2}\selectfont}
\newcommand\FontBcolumn{\fontsize{8}{7.2}\selectfont}
\newcommand\FontCcolumn{\fontsize{10}{7.2}\selectfont}
\newcommand\FontDcolumn{\fontsize{11}{7.2}\selectfont}

\definecolor{gray97}{gray}{.97}
\definecolor{gray75}{gray}{.75}
\definecolor{gray75}{gray}{.45}

\lstdefinestyle{Fortran}{language=[90]Fortran}

\newcommand\FortranStyle
{
\lstset{
frame=Ltb,
framerule=0pt,
columns=fullflexible,
aboveskip=0.5cm,
framextopmargin=3pt,
framexbottommargin=3pt,
framexleftmargin=0.4cm,
framesep=0pt,
rulesep=.4pt,
backgroundcolor=\color{gray97},
rulesepcolor=\color{black},
stringstyle=\ttfamily,
showstringspaces=false,
basicstyle=\ttfamily,
commentstyle=\color{green},
keywordstyle=\color{red},
numbers=left,
numbersep=15pt,
numberstyle=\tiny,
numberfirstline=false,
breaklines=true,
 tabsize=2,
 extendedchars=true,
keepspaces,
}
}

\newcommand\FortranStyleA
{
\lstset{
frame=Ltb,
framerule=0pt,
columns=fullflexible,
aboveskip=0.5cm,
framextopmargin=3pt,
framexbottommargin=3pt,
framexleftmargin=0.4cm,
framesep=0pt,
rulesep=.4pt,
backgroundcolor=\color{gray97},
rulesepcolor=\color{black},
stringstyle=\ttfamily,
showstringspaces=false,
basicstyle=\ttfamily,
commentstyle=\color{green},
keywordstyle=\color{red},
numbersep=15pt,
numberstyle=\tiny,
numberfirstline=false,
breaklines=true,
 tabsize=2,
 extendedchars=true,
keepspaces,
}
}

\newcommand\tab[1][1cm]{\hspace*{#1}}
\newcommand{\light}[1]{\textcolor{lightgray}{#1}}
    
\def\signed #1{{\leavevmode\unskip\nobreak\hfil\penalty50\hskip2em
  \hbox{}\nobreak\hfil(#1)%
  \parfillskip=0pt \finalhyphendemerits=0 \endgraf}}

\newsavebox\mybox
\newenvironment{aquote}[1]
  {\savebox\mybox{#1}\begin{quote}}
  {\signed{\usebox\mybox}\end{quote}}


% items enclosed in square brackets are optional; explanation below
\title{Object Oriented Fortran}
\subtitle{An Overview}
\author{Carlos Cruz}
\institute{
  NASA GSFC Code 606 (ASTG)\\
  Greenbelt, Maryland 20771\\[1ex]
  \texttt{carlos.a.cruz@nasa.gov}
}
\date{October 25, 2018}

\begin{document}

% --- Title page ---
\begin{frame}[plain]
  \titlepage
\end{frame}

\logo{%
  \includegraphics[width=1cm,height=1cm,keepaspectratio]{../../shared/nasa-ball.png}%
  \hspace{\dimexpr\paperwidth-2cm-5pt}%
  \includegraphics[width=1cm,height=1cm,keepaspectratio]{../../shared/ssai-logo.png}%
}


% --- Slide

\begin{frame}{Agenda}

\textcolor{red}{Introduction to OOP}
    \begin{itemize}
        \item History of OOP
        \item Why OOP?
        \item Basic Conceps
        \item A Motivating Example
    \end{itemize}

\end{frame}


% --- Slide ---

\begin{frame}{Motivation}

The basic question we face as computational scientists is: how do we deal with computation?

\end{frame}


% --- Slide ---

\begin{frame}{Programming paradigms}
% a way of conceptualizing what it means to perform computation and how tasks to be carried out on the computer should be structured and organized
\textbf{Functional programming}\\
\textbf{Component based programming}\\
\textbf{Procedural programming}
\begin{itemize}
  \item C, Fortran are procedural programming languages
\begin{itemize}
  \item single program constructed from \textbf{functions} and \textbf{subroutines}
  %emphasis is on doing things so that large programs are divided into smaller programs known as functions.
 \item modularity and re-use achieved through grouping of procedures (\textbf{modules})
  \item data scope based on function scope
 \end{itemize}
 \item generally no explicit link between data and functions
  \item data is accessible and modifiable by functions
\end{itemize}

\end{frame}



% --- Slide

\begin{frame}[fragile]
\frametitle{Conclusion}

\FontCcolumn

\begin{itemize}
\item F2003 includes a solid support of object orientation.
\item Provides opportunities to adopt newer technologies and modernize current earth science models.
\item There is already a F2008 standard, but enhancements are "minor" (submodules, co-arrays)...and it may take a lot longer for compilers to adopt the standard.
\end{itemize}

\bigskip
References:
\begin{itemize}
\item John Reid, "The new features of Fortran 2003, ACM SIGPLAN Fortran Forum 96", 10 (2007)
\item http://www.nag.com/nagware/NP/doc/nag\_f2003.pdf
\item http://www.pgroup.com/doc/pgifortref.pdf
\end{itemize}
\end{frame}

\end{document}