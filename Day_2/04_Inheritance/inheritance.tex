\documentclass[11pt]{beamer}
\setbeamertemplate{navigation symbols}{}
 \setbeamercovered{transparent}
\usepackage{listings}
%\usetheme{Copenhagen}
\usetheme{Singapore}
%\usetheme{Madrid}
%\usetheme{Hannover}
%\usetheme{boxes}
%\usetheme{Boadilla}
\usefonttheme[onlymath]{serif}
\usecolortheme{beaver}
\usepackage{textpos}
\usepackage{fancyvrb}
\usepackage{xcolor}
\usepackage{multicol}
\usepackage{lipsum}
\parskip 1ex

\newcommand\FontAcolumn{\fontsize{6}{7.2}\selectfont}
\newcommand\FontBcolumn{\fontsize{8}{7.2}\selectfont}
\newcommand\FontCcolumn{\fontsize{10}{7.2}\selectfont}
\newcommand\FontDcolumn{\fontsize{11}{7.2}\selectfont}

\definecolor{gray97}{gray}{.97}
\definecolor{gray75}{gray}{.75}
\definecolor{gray75}{gray}{.45}

\lstdefinestyle{Fortran}{language=[90]Fortran}

\newcommand\FortranStyle
{
\lstset{
frame=Ltb,
framerule=0pt,
columns=fullflexible,
aboveskip=0.5cm,
framextopmargin=3pt,
framexbottommargin=3pt,
framexleftmargin=0.4cm,
framesep=0pt,
rulesep=.4pt,
backgroundcolor=\color{gray97},
rulesepcolor=\color{black},
stringstyle=\ttfamily,
showstringspaces=false,
basicstyle=\ttfamily,
commentstyle=\color{green},
keywordstyle=\color{red},
numbers=left,
numbersep=15pt,
numberstyle=\tiny,
numberfirstline=false,
breaklines=true,
 tabsize=2,
 extendedchars=true,
keepspaces,
}
}

\newcommand\FortranStyleA
{
\lstset{
frame=Ltb,
framerule=0pt,
columns=fullflexible,
aboveskip=0.5cm,
framextopmargin=3pt,
framexbottommargin=3pt,
framexleftmargin=0.4cm,
framesep=0pt,
rulesep=.4pt,
backgroundcolor=\color{gray97},
rulesepcolor=\color{black},
stringstyle=\ttfamily,
showstringspaces=false,
basicstyle=\ttfamily,
commentstyle=\color{green},
keywordstyle=\color{red},
numbersep=15pt,
numberstyle=\tiny,
numberfirstline=false,
breaklines=true,
 tabsize=2,
 extendedchars=true,
keepspaces,
}
}

\newcommand\tab[1][1cm]{\hspace*{#1}}
\newcommand{\light}[1]{\textcolor{lightgray}{#1}}
    
\def\signed #1{{\leavevmode\unskip\nobreak\hfil\penalty50\hskip2em
  \hbox{}\nobreak\hfil(#1)%
  \parfillskip=0pt \finalhyphendemerits=0 \endgraf}}

\newsavebox\mybox
\newenvironment{aquote}[1]
  {\savebox\mybox{#1}\begin{quote}}
  {\signed{\usebox\mybox}\end{quote}}


% items enclosed in square brackets are optional; explanation below
\title{Inheritance}
\author{Carlos Cruz}
\institute{
  NASA GSFC Code 606 (ASTG)\\
  Greenbelt, Maryland 20771\\[1ex]
  \texttt{carlos.a.cruz@nasa.gov}
}
\date{October 25, 2018}

\begin{document}

% --- Title page ---
\begin{frame}[plain]
  \titlepage
\end{frame}

\logo{%
  \includegraphics[width=1cm,height=1cm,keepaspectratio]{../../shared/nasa-ball.png}%
  \hspace{\dimexpr\paperwidth-2cm-5pt}%
  \includegraphics[width=1cm,height=1cm,keepaspectratio]{../../shared/ssai-logo.png}%
}


% --- Slide

\begin{frame}{Agenda}

\textcolor{red}{Inheritance}
    \begin{itemize}
    \item Variables as Objects
    \begin{itemize}
    \item Type bound procedures
    \end{itemize}
    \item Type Extension and Inheritance
    \begin{itemize}
    \item Extends attribute
    \item Abstract Types
    \item Abstract Interfaces
    \end{itemize}
    \end{itemize}
  

\end{frame}


% --- Slide ---

\begin{frame}{Type-bound Procedures}

Or "Procedures Bound to a Type by Name"
\begin{itemize}
  \item Allows a Fortran subroutine or procedure to be treated as a method on an object of the given type.\\
  \quad call object\%method(...)\\
  \quad x = object\%func(...)\\
  which is equivalent to\\
  \quad call method(object, ...)\\
  \quad x = func(object, ...)\\
  \item This syntax encourages an object-oriented style of programming that can improve clarity in some contexts.
  \item Procedures can also be bound by type to operators: [=, +, etc. ]
\end{itemize}
 
\end{frame}

% --- Slide

\begin{frame}[fragile]
\frametitle{Type-bound Syntax}

The following example defines a derived type with 2 type bound procedures compute() and retrieve()
\FortranStyle

 \begin{lstlisting}[style=Fortran]
type myType
   real :: value
contains
   procedure :: compute
    procedure :: retrieve
end type myType
\end{lstlisting}
\begin{itemize}
\item Type-bound procedures must be module procedures or external procedures with explicit interfaces.\\
\item By default type-bound names are public, but each entity in a type (including components) can be have a PUBLIC/PRIVATE attribute.
\end{itemize}


\end{frame}

% --- Slide

\begin{frame}[fragile]
\frametitle{Example}


\FortranStyle

\begin{columns}
\begin{column}{0.4\textwidth}
  
 \FontAcolumn

\begin{lstlisting}[style=Fortran]
module myType_mod
   private  ! information hiding
   public :: myType ! except
   type myType
        real :: myValue(4) = 0.0
   contains
        procedure :: write
        procedure :: reset
   end type myType 
contains
   subroutine write (this, unit)
   class(mytype) :: this
      integer, optional :: unit
      if(present(unit)) then
         write (unit, *) this % myValue
      else
         print *, this%myvalue
       endif
   end subroutine write
...
\end{lstlisting}

  \end{column}
  \begin{column}{0.4\textwidth}

 \FontAcolumn
  
\begin{lstlisting}[style=Fortran]
...
subroutine reset(variable)
     class(mytype) :: variable
variable%myvalue = 0.0 end subroutine reset
end module myType_mod
\end{lstlisting}

 \FontBcolumn
Usage:\\
use mytype\_mod\\
type(myType) :: var\\
...\\
call var\%write(unit=6)\\
call var\%reset()\\

  \end{column}
\end{columns}
 

\end{frame}


% --- Slide

\begin{frame}[fragile]
\frametitle{Passed-object dummy arg}

\scriptsize{
By default, type-bound procedures pass the object as the first argument.
\begin{itemize}
  \item Can override behavior with \textcolor{blue}{NOPASS} attribute.\\
  \quad procedure, \textcolor{blue}{NOPASS} :: method\\
  \quad ...\\
  \quad call thing \% method(...)  $\leftarrow$ No object is passed \\
  
  \item Can also specify which argument is to be associated with the passed-object with the \textcolor{blue}{PASS} attribute:\\
  \quad procedure, \textcolor{blue}{PASS}(obj) :: method\\
  \quad ...\\
  \quad subroutine method(x,\textcolor{blue}{obj},y) \\
  \quad ... \\
  \quad call thing \% method(x,y)  $\leftarrow$ Thing is 2nd obj \\
  \item The default can be explicitly confirmed by \\
  \quad procedure, \textcolor{blue}{PASS} :: method\\
\end{itemize}
\textbf{Strongly} recommend that you always use the default.
}

\end{frame}

% --- Slide

\begin{frame}[fragile]
\frametitle{Renaming and Generic}

\begin{itemize}
\item Type-bound procedures can specify an alternative public name using a mechanism analogous to that for the module ONLY clause:\\
  \quad procedure :: write \textcolor{blue}{$=>$} writeInternal\\
  \quad ...\\
  \quad call thing \% method(...)  $\leftarrow$ No object is passed \\
  
  \item Similarly, an external name can be \textbf{overloaded} for multiple interfaces with the GENERIC statement:\\
  \quad type myType\\
  \quad contains\\
  \quad \quad procedure :: addInteger\\
  \quad \quad procedure :: addReal\\
  \quad \quad \textcolor{blue}{generic} :: add $=> $ addReal, addInteger\\
  \quad end type\\
\end{itemize}

\end{frame}


% --- Slide

\begin{frame}[fragile]
\frametitle{Inheritance}
Fortran 2003 introduces OOP inheritance via the EXTENDS attribute for user defined types.
\begin{itemize}
\item Implementation is restricted to single inheritance.
  \begin{itemize}
  \item Inheritance always forms hierarchical trees.
  \end{itemize}

\item Implementation is designed to be efficient such that offsets for components and type-bound procedures can be computed at compile time. ("single lookup")

\end{itemize}
With \emph{type extension}, a developer may add new components and type-bound procedures to an existing derived type even \emph{without} access to the source code for that type.
\end{frame}

% --- Slide

\begin{frame}[fragile]
\frametitle{Inheritance Terminology}

\begin{itemize}
%\item A type is considered to be extensible if it is:
%  \begin{itemize}
%  \item Not a SEQUENCE type
%  \item Not BIND(C)
%  \end{itemize}

\item An extensible type without the EXTENDS attribute is considered to be a 'base type'.
  \begin{itemize}
  \item Base types need not have any components.
  \item Extension need not add any components.
  \end{itemize}
  
\item A type with the EXTENDS attribute is said to be an extended type.
  \begin{itemize}
  \item 'Parent type' is used for the type from which the extension is made.
  \item All the components, and bound procedures of the parent type are inherited by the extended type and they are known by the same names.
  % An extended type can be a parent for yet another type, and so on.
  \end{itemize}
  
\end{itemize}

\end{frame}

% --- Slide

\begin{frame}[fragile]
\frametitle{Syntax for Extends}

\FortranStyle

 \begin{lstlisting}[style=Fortran]
type Location2D
   real :: latitude, longitude
end type Location2D

type, EXTENDS (Location2D) :: Location3D 
   real :: pressureHeight
end type Location3D
   ...
type (Location3D) :: location
lat = location % latitude
lon = location % longitude
height = location % pressureHeight
\end{lstlisting}


\end{frame}

% --- Slide

\begin{frame}[fragile]
\frametitle{The Parent Component}

\begin{itemize}
\item Every extended type has an \emph{implicit component} associated with the parent type
  \begin{itemize}
  \item The component name is the type name of the parent.
  \item Provides \emph{multiple} mechanisms to access components in parent type
  \end{itemize}
\end{itemize}

From the previous example we could do:
\FortranStyle
 \begin{lstlisting}[style=Fortran]
type (Location2D) :: latLon
latLon = location % Location2D
lat = location % Location2D % latitude
\end{lstlisting}

\end{frame}

% --- Slide

\begin{frame}[fragile]
\frametitle{Extends and Type-bound}

\begin{itemize}
\item Type-bound procedures in the parent may be invoked within extended types.
\item Extended types may add additional type-bound procedures in the natural fashion.
\item An extended type can \emph{override} a type-bound procedure in the parent - specifying new behavior in the extended type.
  \begin{itemize}
  \item The keyword \textcolor{blue}{NON\_OVERRIDABLE} can be used to prohibit extended classes from overriding behavior:\\
  \quad \emph{procedure, \textcolor{blue}{NON\_OVERRIDABLE} :: foo}
  \end{itemize}
\end{itemize}


\end{frame}

% --- Slide

\begin{frame}[fragile]
\frametitle{Overriding Example}

\scriptsize{
\FortranStyle
 \begin{lstlisting}[style=Fortran]
type square
   real :: length
contains
   procedure :: area => square_area
end type square

type, extends(square) :: rectangle  ! inherits area
   real :: width
contains
   procedure :: area => rectangle_area  ! overriding area
end type rectangle

real function square_area(this)
   square_area = (this % length) ** 2

real function rectangle_area(this)
   rectangle_area = (this % length) * (this % width)
\end{lstlisting}
}

\end{frame}

% --- Slide

\begin{frame}[fragile]
\frametitle{Abstract Types}

\begin{itemize}
\item It is often useful to have a base type that declares methods (type-bound procedures) that are not implemented except in extended classes.
\item Fortran 2003 uses the \textcolor{blue}{ABSTRACT} attribute to denote such a type.
  \begin{itemize}
  \item The \textcolor{blue}{DEFERRED} attribute is used for those methods which are not to be implemented.
  \item No variables can be declared to be of an abstract type.
  \end{itemize}
\end{itemize}

\end{frame}

% --- Slide

\begin{frame}[fragile]
\frametitle{Abstract Example}

\scriptsize{
\FortranStyle
 \begin{lstlisting}[style=Fortran]
type, ABSTRACT :: abstract_shape
contains
   procedure (area_interface), DEFERRED :: area 
end type abstract_shape
...
ABSTRACT interface
   subroutine area_interface(obj)
      import abstract_shape
      class (abstract_shape) :: obj
   end subroutine area_interface
end interface
...
type, EXTENDS(abstract_shape) :: square
   real :: length
contains
   procedure :: area => square_area !  Provide concrete
end type square
\end{lstlisting}
}

\end{frame}

% --- Slide

\begin{frame}[fragile]
\frametitle{Shape Class Exercise}

Let's look at the shape class files. Then
\begin{itemize}
\item Create a new file triangle\_mod.F90 that contains
\bi
\item a constructor that creates a triangle object.
\item a function that calculates the area of a triangle.
\ei
\item edit test\_shapes.F90 and add code to print the result
 \end{itemize}
Then build the executable and run the code using Makefile_exercise.

\end{frame}

% --- Slide ---
\begin{frame}[fragile]
\frametitle{}


\centering{
\huge{
Shape Class
}
}

\end{frame}


% --- Slide

\begin{frame}[fragile]
\frametitle{Conclusion}

\end{frame}

\end{document}