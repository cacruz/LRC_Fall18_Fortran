\documentclass[11pt]{beamer}
\setbeamertemplate{navigation symbols}{}
 \setbeamercovered{transparent}
\usepackage{listings}
%\usetheme{Copenhagen}
\usetheme{Singapore}
%\usetheme{Madrid}
%\usetheme{Hannover}
%\usetheme{boxes}
%\usetheme{Boadilla}
\usefonttheme[onlymath]{serif}
\usecolortheme{beaver}
\usepackage{textpos}
\usepackage{fancyvrb}
\usepackage{xcolor}
\usepackage{multicol}
\usepackage{lipsum}
\parskip 1ex

\newcommand\FontAcolumn{\fontsize{6}{7.2}\selectfont}
\newcommand\FontBcolumn{\fontsize{8}{7.2}\selectfont}
\newcommand\FontCcolumn{\fontsize{10}{7.2}\selectfont}
\newcommand\FontDcolumn{\fontsize{11}{7.2}\selectfont}

\definecolor{gray97}{gray}{.97}
\definecolor{gray75}{gray}{.75}
\definecolor{gray75}{gray}{.45}

\lstdefinestyle{Fortran}{language=[90]Fortran}

\newcommand\FortranStyle
{
\lstset{
frame=Ltb,
framerule=0pt,
columns=fullflexible,
aboveskip=0.5cm,
framextopmargin=3pt,
framexbottommargin=3pt,
framexleftmargin=0.4cm,
framesep=0pt,
rulesep=.4pt,
backgroundcolor=\color{gray97},
rulesepcolor=\color{black},
stringstyle=\ttfamily,
showstringspaces=false,
basicstyle=\ttfamily,
commentstyle=\color{green},
keywordstyle=\color{red},
numbers=left,
numbersep=15pt,
numberstyle=\tiny,
numberfirstline=false,
breaklines=true,
 tabsize=2,
 extendedchars=true,
keepspaces,
}
}

\newcommand\FortranStyleA
{
\lstset{
frame=Ltb,
framerule=0pt,
columns=fullflexible,
aboveskip=0.5cm,
framextopmargin=3pt,
framexbottommargin=3pt,
framexleftmargin=0.4cm,
framesep=0pt,
rulesep=.4pt,
backgroundcolor=\color{gray97},
rulesepcolor=\color{black},
stringstyle=\ttfamily,
showstringspaces=false,
basicstyle=\ttfamily,
commentstyle=\color{green},
keywordstyle=\color{red},
numbersep=15pt,
numberstyle=\tiny,
numberfirstline=false,
breaklines=true,
 tabsize=2,
 extendedchars=true,
keepspaces,
}
}

\newcommand\tab[1][1cm]{\hspace*{#1}}
\newcommand{\light}[1]{\textcolor{lightgray}{#1}}
    
\def\signed #1{{\leavevmode\unskip\nobreak\hfil\penalty50\hskip2em
  \hbox{}\nobreak\hfil(#1)%
  \parfillskip=0pt \finalhyphendemerits=0 \endgraf}}

\newsavebox\mybox
\newenvironment{aquote}[1]
  {\savebox\mybox{#1}\begin{quote}}
  {\signed{\usebox\mybox}\end{quote}}


% items enclosed in square brackets are optional; explanation below
\title{Miscellaneous Items}
\author{Carlos Cruz}
\institute{
  NASA GSFC Code 606 (ASTG)\\
  Greenbelt, Maryland 20771\\[1ex]
  \texttt{carlos.a.cruz@nasa.gov}
}
\date{October 25, 2018}

\begin{document}

% --- Title page ---
\begin{frame}[plain]
  \titlepage
\end{frame}

\logo{%
  \includegraphics[width=1cm,height=1cm,keepaspectratio]{../../shared/nasa-ball.png}%
  \hspace{\dimexpr\paperwidth-2cm-5pt}%
  \includegraphics[width=1cm,height=1cm,keepaspectratio]{../../shared/ssai-logo.png}%
}


% --- Slide

\begin{frame}{Agenda}

\textcolor{red}{Miscellaneous Items}
    \begin{itemize}
    \item Computing Environment
    \item Array Constructor Syntax
    \item Module Enhancements
    \begin{itemize}
    \item IMPORT
    \item New Attributes
    \item Renaming Operatos
    \end{itemize}
    \item Changes to Intrinsic Functions
    \item Complex Constants
    \item Support for international character sets
    \end{itemize}
  

\end{frame}


% --- Slide ---

\begin{frame}{Computing Environment}
\begin{itemize}
  \item From the intrinsic module ISO\_FORTRAN\_ENV
  \item For the following assume we have launched the executable with the command line: \\
  \quad \$ foo.x apple 5 z
  \item COMMAND\_ARGUMENT\_COUNT()
  \begin{itemize}
  \item Returns integer number of command arguments
  \item Example command returns 3
  \end{itemize}
 
 \item GET\_COMMAND([COMMAND,LENGTH,STATUS])
  \begin{itemize}
  \item All INTENT(OUT)andOPTIONAL
  \item LENGTH - integer \# of characters in command
  \item STATUS - integer (success/failure)
  \item Results for example command:
  \begin{itemize}
  \item COMMAND = "foo.x apple 5 z"
  \item LENGTH=15
  \end{itemize}
 \end{itemize}

\end{itemize}

\end{frame}


% --- Slide ---

\begin{frame}{Computing Environment}
\begin{itemize}
 \item GET\_COMMAND\_ARGUMENT(NUMBER[,VALUE,LENGT H,STATUS])
  \begin{itemize}
  \item NUMBER - selects argument
  \item VALUE - character, intent(out) value of argument
  \item LENGTH - number of characters in argument
  \item STATUS - integer (success/failure)
  \item Example command yields:
  \begin{itemize}
  \item GET\_COMMAND\_ARGUMENT(0,VALUE,LENGTH) yields \\
  \quad VALUE="foo.x", LENGTH=5
  \item GET\_COMMAND\_ARGUMENT(2,VALUE,LENGTH) yields \\
  \quad VALUE="5", LENGTH=1
  \end{itemize}
 \end{itemize}

\end{itemize}

\end{frame}

% --- Slide ---

\begin{frame}[fragile]
\frametitle{Environment examples}

\scriptsize{
Getting command arguments:
\FortranStyle
\begin{lstlisting}[style=Fortran]
use ISO_FORTRAN_ENV 
character(len=MAXLEN_ARG) :: arg1, arg2 
call get_command_argument(1, VALUE=arg1) 
call get_command_argument(2, VALUE=arg2) 
read(arg1,'(i)') nx
read(arg2,'(i)') ny
\end{lstlisting}
Getting an environment variable::
\begin{lstlisting}[style=Fortran]
use ISO_FORTRAN_ENV 
character(len=100) :: myShell
call get_environment_variable('SHELL', myShell)
\end{lstlisting}
}

\end{frame}



% --- Slide

\begin{frame}[fragile]
\frametitle{Conclusion}

\end{frame}

\end{document}