\documentclass[11pt]{beamer}
\setbeamertemplate{navigation symbols}{}
 \setbeamercovered{transparent}
\usepackage{listings}
%\usetheme{Copenhagen}
\usetheme{Singapore}
%\usetheme{Madrid}
%\usetheme{Hannover}
%\usetheme{boxes}
%\usetheme{Boadilla}
\usefonttheme[onlymath]{serif}
\usecolortheme{beaver}
\usepackage{textpos}
\usepackage{fancyvrb}
\usepackage{xcolor}
\usepackage{multicol}
\usepackage{lipsum}
\parskip 1ex

\newcommand\FontAcolumn{\fontsize{6}{7.2}\selectfont}
\newcommand\FontBcolumn{\fontsize{8}{7.2}\selectfont}
\newcommand\FontCcolumn{\fontsize{10}{7.2}\selectfont}
\newcommand\FontDcolumn{\fontsize{11}{7.2}\selectfont}

\definecolor{gray97}{gray}{.97}
\definecolor{gray75}{gray}{.75}
\definecolor{gray75}{gray}{.45}

\lstdefinestyle{Fortran}{language=[90]Fortran}

\newcommand\FortranStyle
{
\lstset{
frame=Ltb,
framerule=0pt,
columns=fullflexible,
aboveskip=0.5cm,
framextopmargin=3pt,
framexbottommargin=3pt,
framexleftmargin=0.4cm,
framesep=0pt,
rulesep=.4pt,
backgroundcolor=\color{gray97},
rulesepcolor=\color{black},
stringstyle=\ttfamily,
showstringspaces=false,
basicstyle=\ttfamily,
commentstyle=\color{green},
keywordstyle=\color{red},
numbers=left,
numbersep=15pt,
numberstyle=\tiny,
numberfirstline=false,
breaklines=true,
 tabsize=2,
 extendedchars=true,
keepspaces,
}
}

\newcommand\FortranStyleA
{
\lstset{
frame=Ltb,
framerule=0pt,
columns=fullflexible,
aboveskip=0.5cm,
framextopmargin=3pt,
framexbottommargin=3pt,
framexleftmargin=0.4cm,
framesep=0pt,
rulesep=.4pt,
backgroundcolor=\color{gray97},
rulesepcolor=\color{black},
stringstyle=\ttfamily,
showstringspaces=false,
basicstyle=\ttfamily,
commentstyle=\color{green},
keywordstyle=\color{red},
numbersep=15pt,
numberstyle=\tiny,
numberfirstline=false,
breaklines=true,
 tabsize=2,
 extendedchars=true,
keepspaces,
}
}

\newcommand\tab[1][1cm]{\hspace*{#1}}
\newcommand{\light}[1]{\textcolor{lightgray}{#1}}
    
\def\signed #1{{\leavevmode\unskip\nobreak\hfil\penalty50\hskip2em
  \hbox{}\nobreak\hfil(#1)%
  \parfillskip=0pt \finalhyphendemerits=0 \endgraf}}

\newsavebox\mybox
\newenvironment{aquote}[1]
  {\savebox\mybox{#1}\begin{quote}}
  {\signed{\usebox\mybox}\end{quote}}


% items enclosed in square brackets are optional; explanation below
\title{Miscellaneous Items}
\author{Carlos Cruz}
\institute{
  NASA GSFC Code 606 (ASTG)\\
  Greenbelt, Maryland 20771\\[1ex]
  \texttt{carlos.a.cruz@nasa.gov}
}
\date{October 25, 2018}

\begin{document}

% --- Title page ---
\begin{frame}[plain]
  \titlepage
\end{frame}

\logo{%
  \includegraphics[width=1cm,height=1cm,keepaspectratio]{../../shared/nasa-ball.png}%
  \hspace{\dimexpr\paperwidth-2cm-5pt}%
  \includegraphics[width=1cm,height=1cm,keepaspectratio]{../../shared/ssai-logo.png}%
}


% --- Slide

\begin{frame}{Agenda}

\textcolor{red}{Miscellaneous Items}
    \begin{itemize}
    \item Computing Environment
    \item Array Constructor Syntax
    \item Module Enhancements
    \begin{itemize}
    \item IMPORT
    \item New Attributes
    \item Renaming Operatos
    \end{itemize}
    \item Changes to Intrinsic Functions
    \item Complex Constants
    \end{itemize}
  

\end{frame}


% --- Slide ---

\begin{frame}{Computing Environment}
\begin{itemize}
  \item From the intrinsic module ISO\_FORTRAN\_ENV
  \item For the following assume we have launched the executable with the command line: \\
  \quad \$ foo.x apple 5 z
  \item COMMAND\_ARGUMENT\_COUNT()
  \begin{itemize}
  \item Returns integer number of command arguments
  \item Example command returns 3
  \end{itemize}
 
 \item GET\_COMMAND([COMMAND,LENGTH,STATUS])
  \begin{itemize}
  \item All INTENT(OUT)andOPTIONAL
  \item LENGTH - integer \# of characters in command
  \item STATUS - integer (success/failure)
  \item Results for example command:
  \begin{itemize}
  \item COMMAND = "foo.x apple 5 z"
  \item LENGTH=15
  \end{itemize}
 \end{itemize}

\end{itemize}

\end{frame}


% --- Slide ---

\begin{frame}{Computing Environment}
\begin{itemize}
 \item GET\_COMMAND\_ARGUMENT(NUMBER[,VALUE,LENGT H,STATUS])
  \begin{itemize}
  \item NUMBER - selects argument
  \item VALUE - character, intent(out) value of argument
  \item LENGTH - number of characters in argument
  \item STATUS - integer (success/failure)
  \item Example command yields:
  \begin{itemize}
  \item GET\_COMMAND\_ARGUMENT(0,VALUE,LENGTH) yields \\
  \quad VALUE="foo.x", LENGTH=5
  \item GET\_COMMAND\_ARGUMENT(2,VALUE,LENGTH) yields \\
  \quad VALUE="5", LENGTH=1
  \end{itemize}
 \end{itemize}

\end{itemize}

\end{frame}

% --- Slide ---

\begin{frame}[fragile]
\frametitle{Environment examples}

\scriptsize{
Getting command arguments:
\FortranStyle
\begin{lstlisting}[style=Fortran]
use ISO_FORTRAN_ENV 
character(len=MAXLEN_ARG) :: arg1, arg2 
call get_command_argument(1, VALUE=arg1) 
call get_command_argument(2, VALUE=arg2) 
read(arg1,'(i)') nx
read(arg2,'(i)') ny
\end{lstlisting}
Getting an environment variable::
\begin{lstlisting}[style=Fortran]
use ISO_FORTRAN_ENV 
character(len=100) :: myShell
call get_environment_variable('SHELL', myShell)
\end{lstlisting}
}

\end{frame}


% --- Slide ---

\begin{frame}{Array Constructor}

\begin{itemize}
 \item Can now use "[" and "]" rather than "(/", "/)" to construct arrays:\\
 \quad x(1:5) = [0.,1.,2.,3.,4.]
  \item Can also specify type \textbf{inside} constructor
  \begin{itemize}
  \item VALUE - character, intent(out) value of argument
  \item LENGTH - number of characters in argument
  \item STATUS - integer (success/failure)
  \item Example command yields:
  \begin{itemize}
  \item GET\_COMMAND\_ARGUMENT(0,VALUE,LENGTH) yields \\
  \quad VALUE="foo.x", LENGTH=5
  \item GET\_COMMAND\_ARGUMENT(2,VALUE,LENGTH) yields \\
  \quad VALUE="5", LENGTH=1
  \end{itemize}
 \end{itemize}

\end{itemize}

\end{frame}


% --- Slide ---

\begin{frame}[fragile]
\frametitle{IMPORT statement}

\scriptsize{
A common pitfall when using F90/F95 is the declaration of an interface block than needs to "use" a derived type defined in the same module:
\FortranStyle
\begin{lstlisting}[style=Fortran]
module foo
   type bar
      integer :: I,J
   end type bar
   interface
      subroutine externFunc(B)
         use foo, only: bar ! Not allowed?
         type (bar) :: B
      end subroutine externFunc
  end interface
...
\end{lstlisting}
}

\end{frame}

% --- Slide ---

\begin{frame}[fragile]
\frametitle{IMPORT statement}

\scriptsize{
IMPORT is a new statement to address this issue.
  \begin{itemize}
  \item Very similar to USE statement.
  \item Specifies all entities in host scoping unit that are accessible
  \item \emph{Only} allowed in an interface body within a module
  \end{itemize}
 Exmaple:
\FortranStyle
\begin{lstlisting}[style=Fortran]
...
interface
     subroutine externFunc(B)
        import foo, only: bar
        type (bar) :: B
     end subroutine externFunc
  end interface
\end{lstlisting}
}

\end{frame}

% --- Slide ---

\begin{frame}[fragile]
\frametitle{PROTECTED attribute}

\scriptsize{
F2003 introduces the new attribute PROTECTED which provides a safety mechanism analogous to INTENT(IN)
  \begin{itemize}
  \item Specifies that the variable (or pointer status) may be altered only within the host module.
  \item Property is recursive. I.e. if a variable of derived type is PROTECTED, all of its sub-objects also have the attribute
  \item For pointers, only the association status is protected. The target may be modified elsewhere.
  \end{itemize}
 Example:
\FortranStyle
\begin{lstlisting}[style=Fortran]
module foo
private ! Good default
real, public :: pi
protected :: pi ! Allow value to be read
...
\end{lstlisting}
}

\end{frame}

% --- Slide ---

\begin{frame}[fragile]
\frametitle{VOLATILE attribute}

\begin{itemize}
 \item Introduced for a data object to indicate that its value might be modified by means external to the program.
  \begin{itemize}
  \item Non standard extensions (e.g. threads)
  \item Card connected to external lab instrument
  \end{itemize}
  
  \item Effect is that the compiler is required to not rely on values in cache or other temporary memory.
  \item If an object has the VOLATILE attribute, so do all of its subobjects.
  \item For pointers, attribute refers only to the association status, \emph{not} the target.
\end{itemize}

\end{frame}


% --- Slide ---

\begin{frame}[fragile]
\frametitle{Renaming operators}

\begin{itemize}
 \item F2003 extends the rename capability on USE statements to include renaming operators that are not intrinsic operators:\\
 \FortranStyle
\begin{lstlisting}[style=Fortran]
USE a_mod, OPERATOR(.MyAdd.) => OPERATOR(.ADD.)
\end{lstlisting}
 \item This allows .MyAdd. to denote the operator .ADD. accessed from the module.
 \end{itemize}

\end{frame}


% --- Slide ---

\begin{frame}[fragile]
\frametitle{Changes to Intrinsic Functions}

\begin{itemize}
 \item Argument COUNT\_RATE for SYSTEM\_CLOCK() can now be of type real.
 \begin{itemize}
 \item Previously had to convert integer to compute reciprocal to determine elapsed time
 \end{itemize}
 \item  MAX, MAXLOC, MAXVAL, MIN, MINLOC, MINVAL have all been extend to apply to type CHARACTER
 \item ATAN2, LOG, and SQRT have minor changes to take into account positive/negative zero for vendors that support the distinction.
 \end{itemize}

\end{frame}

% --- Slide ---

\begin{frame}[fragile]
\frametitle{Lengths of Names/Constants}

\begin{itemize}
 \item Variables may be declared with names of up to 63 characters
 \item Statements of up to 256 lines are permitted.
 \item Primarily aimed at supporting automatic code generation
 \end{itemize}

\end{frame}

% --- Slide ---

\begin{frame}[fragile]
\frametitle{Complex Constants}

Named constants may be used to specify real or imaginary parts of a complex constant:
 \FortranStyle
\begin{lstlisting}[style=Fortran]
REAL, PARAMETER :: pi = 3.1415926535897932384 
COMPLEX :: C = (0.0,pi)
\end{lstlisting}

\end{frame}


\end{document}
